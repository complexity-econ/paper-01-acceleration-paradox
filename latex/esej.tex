\documentclass[12pt,a4paper]{article}

% Font and language setup
\usepackage{fontspec}
\usepackage{polyglossia}
\setmainlanguage{polish}
\setotherlanguage{english}

% Page geometry
\usepackage[margin=2.5cm]{geometry}

% Math packages
\usepackage{amsmath}
\usepackage{amssymb}

% Graphics and figures
\usepackage{graphicx}
\usepackage{float}
\usepackage{chngcntr}
\graphicspath{{figures/}}

% Tables
\usepackage{booktabs}
\usepackage{tabularx}

% Hyperlinks
\usepackage{hyperref}
\hypersetup{
    colorlinks=true,
    linkcolor=black,
    citecolor=black,
    urlcolor=blue
}

% Bibliography
\usepackage[backend=bibtex,style=authoryear,sorting=nyt]{biblatex}
\addbibresource{references.bib}

% Line spacing
\usepackage{setspace}
\onehalfspacing

% Section formatting
\usepackage{titlesec}
\setcounter{secnumdepth}{0}  % No automatic numbering in TOC/headings
\setcounter{tocdepth}{4}     % Show up to paragraph level in TOC
\titleformat{\section}{\Large\bfseries}{}{0em}{}
\titlespacing*{\section}{0pt}{0pt}{1em}
\newcommand{\sectionbreak}{\clearpage}
\titleformat{\subsection}{\large\bfseries}{}{0em}{}
\titleformat{\subsubsection}{\normalsize\bfseries}{}{0em}{}
\titleformat{\paragraph}[block]{\normalsize\bfseries}{}{0em}{}
% Manual chapter counter for figure/table numbering
\newcounter{mychapter}
\renewcommand{\thefigure}{\arabic{mychapter}.\arabic{figure}}
\renewcommand{\thetable}{\arabic{mychapter}.\arabic{table}}

% Headers and footers
\usepackage{fancyhdr}
\pagestyle{fancy}
\fancyhf{}
\fancyhead[L]{Paradoks Akceleracji: BDP jako katalizator automatyzacji}
\fancyhead[R]{\thepage}
\renewcommand{\headrulewidth}{0.4pt}

% Document metadata
\title{\textbf{Paradoks Akceleracji: Bezwarunkowy Dochód Podstawowy jako Katalizator Automatyzacji w Małej Gospodarce Otwartej}}
\author{Mateusz Maciaszek}
\date{2026}

\begin{document}

\maketitle

\begin{abstract}
\noindent\textbf{Abstrakt:}

Niniejsza praca podejmuje krytyczną analizę wpływu Bezwarunkowego Dochodu Podstawowego (BDP) na procesy decyzyjne w przedsiębiorstwach oraz stabilność makroekonomiczną w warunkach małej gospodarki otwartej (Polska). Kwestionując klasyczny paradygmat fiskalny zaprezentowany m.in. w raporcie MFW \emph{„Fiscal Monitor: Tackling Inequality''} \cite{imf2017}, który traktuje technologię jako czynnik egzogeniczny, praca stawia tezę o istnieniu \textbf{„Paradoksu Akceleracji''}. Zgodnie z tą hipotezą, BDP nie jest skutkiem bezrobocia technologicznego, lecz jego pierwotną przyczyną (katalizatorem), wymuszającą na przedsiębiorstwach skokową substytucję pracy kapitałem.

Metodologia badawcza opiera się na syntezie Nowoczesnej Teorii Monetarnej (MMT), ekonomii złożoności (nieergodyczności) oraz zmodyfikowanej funkcji produkcji CES. Przeprowadzona symulacja bilansowa wykazuje, że w obliczu szoku kosztowego (wzrost płacy rezerwowej) i monetarnego (wzrost stóp procentowych w reakcji na inflację), tradycyjne modele biznesowe napotykają \textbf{barierę pochłaniającą} (ryzyko bankructwa). Jedyną strategią przetrwania staje się nieliniowa ucieczka w aktywa o zerowym koszcie krańcowym (Sztuczna Inteligencja).

Praca dowodzi, że wysoki koszt pieniądza działa w tym modelu jako filtr ewolucyjny, przyspieszający dyfuzję innowacji. Zidentyfikowano również mechanizm \textbf{„Endogenicznej Deflacji Technologicznej''}, w którym skokowy wzrost produktywności AI neutralizuje inflacyjny impuls popytowy BDP. Analiza kontrfaktyczna wskazuje jednak, że powodzenie tej transformacji jest ściśle uzależnione od posiadania suwerennej waluty. W reżimie strefy Euro, pozbawionym autonomicznego mechanizmu stóp procentowych, wprowadzenie BDP prowadzi do ryzyka niewypłacalności państwa, a nie do modernizacji technologicznej.

Wyniki zostały zweryfikowane za pomocą symulacji agentowej SFC-ABM (Stock-Flow Consistent Agent-Based Model) obejmującej 10 000 heterogenicznych firm w 6 sektorach (kalibracja GUS 2024), połączonych siecią Wattsa-Strogatz, uruchomionej 100 razy (Monte Carlo) na 3 scenariuszach makroekonomicznych. Symulacja potwierdza emergentną dynamikę: S-krzywą adopcji technologii, szczyt inflacji 13,7\% opanowany regułą Taylora oraz endogeniczną deflację technologiczną w fazie końcowej. Analiza wrażliwości ujawnia nieliniową relację w kształcie odwróconego U: BDP = 2 000 PLN stanowi punkt krytyczny transformacji technologicznej (adopcja 61,9\% $\pm$ 16,4\%, rozkład bimodalny), podczas gdy zarówno brak BDP (12,9\%), jak i nadmierny BDP = 3 000 PLN (32,8\%) prowadzą do suboptymalnych wyników. Kalibracja modelu do danych GUS/NBP 2024 (6 sektorów, sektorowe mnożniki płacowe) ujawnia Podwójny Paradoks Akceleracji: BDP jednocześnie przyspiesza automatyzację w sektorach o wysokiej elastyczności substytucji (BPO/SSC: +21pp), a hamuje ją w przemyśle (−13pp) przez kanał kredytowy --- przy czym 75\% gospodarki (usługi, sektor publiczny, rolnictwo) pozostaje odporne na transformację niezależnie od poziomu BDP.

\textbf{Słowa kluczowe:} Bezwarunkowy Dochód Podstawowy (BDP), Paradoks Akceleracji, SFC-ABM, Bifurkacja, Ekonomia Nieergodyczna, MMT, Funkcja Produkcji CES, Automatyzacja, Deflacja Technologiczna, Polityka Monetarna, Strefa Euro.
\end{abstract}

\newpage
\tableofcontents
\newpage

\setcounter{mychapter}{0}\setcounter{figure}{0}\setcounter{table}{0}
\section{WSTĘP}
\label{wstep}

\subsection{Kontekst Badawczy i Zarys Problemu}
\label{kontekst-badawczy-i-zarys-problemu}

Współczesny dyskurs makroekonomiczny zdominowany jest przez paradygmat „technologicznego bezrobocia'', spopularyzowany m.in. przez prace Brynjolfssona i McAfee \cite{brynjolfsson2014}. W tej narracji, rosnąca rola sztucznej inteligencji (AI) i robotyzacji jest traktowana jako \emph{zmienna niezależna} -- pierwotna siła, która wypiera pracę ludzką, rodząc konieczność wprowadzenia Bezwarunkowego Dochodu Podstawowego (BDP) jako \emph{zmiennej zależnej} (instrumentu osłonowego). Instytucje takie jak Międzynarodowy Fundusz Walutowy (MFW) w swoich raportach (Fiscal Monitor 2017 \cite{imf2017}) analizują BDP głównie przez pryzmat kosztów fiskalnych i dylematów redystrybucyjnych, zakładając, że struktura produkcji jest względnie stała, a rynek pracy dąży do równowagi w długim okresie.

Niniejsza praca stawia tezę fundamentalnie odwrotną, kwestionując linearność powyższego rozumowania. Postuluję istnienie \textbf{„Paradoksu Akceleracji''}, w którym wprowadzenie BDP nie jest skutkiem, lecz \emph{pra-przyczyną} (katalizatorem) gwałtownej transformacji technologicznej przedsiębiorstw.

Opierając się na analizie kosztu pieniądza w gospodarce otwartej oraz teorii płacy rezerwowej, praca dowodzi, że szok monetarny i popytowy wywołany przez BDP uruchamia mechanizm sprzężenia zwrotnego, który zmusza menedżerów do „ucieczki do przodu'' -- w stronę aktywów o zerowym koszcie krańcowym (AI). Dzieje się tak, ponieważ BDP drastycznie zmienia parametry wejściowe funkcji produkcji: podnosi koszt kapitału ludzkiego (presja płacowa) oraz koszt kapitału finansowego (reakcja stóp procentowych na inflację), czyniąc tradycyjne modele biznesowe nierentownymi.

\subsection{Luka w Literaturze: Błąd Ergodyczności}
\label{luka-w-literaturze-blad-ergodycznosci}

Kluczową słabością dotychczasowych analiz (w tym modeli DSGE stosowanych przez MFW) jest założenie \textbf{ergodyczności} procesów gospodarczych. Modele te zakładają, że rynki są układami dążącymi do średniej, a szoki (takie jak wprowadzenie BDP) są symetryczne i odwracalne w czasie. Zakłada się, że przedsiębiorstwo może przetrwać okres przejściowy, optymalizując zyski w nieskończonym horyzoncie czasowym.

W niniejszym opracowaniu odrzucam to założenie na rzecz \textbf{Ekonomii Złożoności (Complexity Economics)} i koncepcji nieergodyczności rynków finansowych \cite{peters2019}. W rzeczywistym świecie menedżerskim czas jest zasobem krytycznym, a ścieżki rozwoju są zależne od historii (path dependence). Wprowadzenie BDP traktuję jako zmianę parametru kontrolnego w nieliniowym układzie dynamicznym, co prowadzi do \textbf{bifurkacji}. Przedsiębiorstwa napotykają tzw. \textbf{barierę pochłaniającą (absorbing barrier)} -- punkt, w którym płynność finansowa kończy się przed osiągnięciem nowej równowagi.

W obliczu „szoku BDP'', menedżer nie ma luksusu oczekiwania na powrót rynku do równowagi. Staje przed binarnym wyborem: bankructwo (uderzenie w barierę) lub natychmiastowa, skokowa substytucja pracy drogimi, ale efektywnymi algorytmami AI. Ta decyzja ma charakter histerezy -- raz dokonana automatyzacja jest nieodwracalna, nawet jeśli w przyszłości parametry makroekonomiczne (stopy procentowe, podaż pracy) powrócą do normy.

\subsection{Metodologia: Synteza MMT i Analizy Menedżerskiej}
\label{metodologia-synteza-mmt-i-analizy-menedzerskiej}

Aby udowodnić powyższą tezę, praca wykorzystuje unikalną syntezę dwóch odległych nurtów:

\begin{enumerate}
\item \textbf{Nowoczesnej Teorii Monetarnej (MMT) w ujęciu sald sektorowych:} Analizuję przepływy między bilansem banku centralnego, sektorem prywatnym a rynkiem długu \cite{godley2007,wray2015}. Wskazuję, że w gospodarce otwartej (Polska) kreacja pieniądza na BDP musi spotkać się z barierą realnych zasobów, co wymusza reakcję stóp procentowych.
\item \textbf{Mikroekonomicznej teorii decyzji (Funkcja produkcji CES):} Modeluję decyzję menedżera jako wybór między kapitałem ($K$) a pracą ($L$) w warunkach, gdy elastyczność substytucji dla AI dąży do nieskończoności, a koszt długu ($r$) rośnie.
\end{enumerate}

\subsection{Główny Wkład Pracy: AI jako Deflator w Modelu MMT}
\label{glowny-wklad-pracy-ai-jako-deflator-w-modelu-mmt}

Najważniejszym wkładem teoretycznym niniejszej pracy jest propozycja rozwiązania głównego dylematu MMT. Tradycyjna krytyka BDP wskazuje, że bez drastycznych podwyżek podatków (ściągających pieniądz z rynku), program ten doprowadzi do hiperinflacji.

Praca wprowadza koncepcję \textbf{„Wymuszonej Deflacji Technologicznej''}. Dowodzę, że masowe, wymuszone przez wysoki koszt pracy i pieniądza inwestycje w AI prowadzą do skokowego wzrostu produktywności przy niemal zerowym koszcie krańcowym. W efekcie, sztuczna inteligencja przejmuje w modelu makroekonomicznym rolę podatków -- „absorbuje'' nadmiarowy popyt wykreowany przez BDP poprzez gigantyczną podaż tanich usług i dóbr. Tym samym, paradoksalnie, to drogi pieniądz i roszczeniowy rynek pracy są warunkiem koniecznym do utrzymania stabilności waluty w erze BDP.

W kolejnych rozdziałach przedstawiony zostanie formalny model matematyczny tego zjawiska (Rozdział 3), analiza bilansowa T-kont i studium przypadku (sekcje 4.1--4.6), symulacja agentowa SFC-ABM z analizą Monte Carlo (sekcje 4.7--4.8), analiza dobrostanu (sekcja 4.9) oraz kalibracja strukturalna do danych GUS 2024 (sekcja 4.10) dla przedsiębiorstwa działającego w warunkach polskiej gospodarki otwartej.

\setcounter{mychapter}{1}\setcounter{figure}{0}\setcounter{table}{0}
\section{ROZDZIAŁ 1}
\label{rozdzial-1}

\subsection{Stan gry -- Ograniczenia klasycznych modeli makroekonomicznych w obliczu szoku BDP}
\label{stan-gry-ograniczenia-klasycznych-modeli}

Celem niniejszego rozdziału jest krytyczna analiza dotychczasowego dorobku literatury przedmiotu w zakresie makroekonomicznych skutków wprowadzenia Bezwarunkowego Dochodu Podstawowego (BDP), ze szczególnym uwzględnieniem perspektywy fiskalnej prezentowanej przez Międzynarodowy Fundusz Walutowy. W dalszej części rozdziału analiza zostanie osadzona w realiach polskiej gospodarki otwartej, wskazując na mechanizm transmisji impulsu popytowego na koszt pieniądza. Rozdział kończy się zidentyfikowaniem luki badawczej wynikającej z przyjęcia błędnego założenia o ergodyczności procesów rynkowych w klasycznych modelach równowagi ogólnej.

\subsubsection{1.1. Dylemat fiskalny w ujęciu neoklasycznym: Analiza raportu MFW (2017)}
\label{dylemat-fiskalny-w-ujeciu-neoklasycznym}

Punktem wyjścia do analizy jest raport Międzynarodowego Funduszu Walutowego \emph{„Fiscal Monitor: Tackling Inequality''} \cite{imf2017}, który stanowi kanon myślenia o BDP w kategoriach głównego nurtu ekonomii. MFW podchodzi do zagadnienia dochodu gwarantowanego przez pryzmat klasycznego dylematu \emph{efficiency-equity trade-off} (kompromisu między efektywnością a równością).

Według szacunków Funduszu, koszt fiskalny wprowadzenia BDP na poziomie zaledwie 25\% mediany dochodu per capita wynosiłby w gospodarkach rozwiniętych średnio 6-7\% PKB. Tak gigantyczna relokacja środków publicznych wymagałaby drastycznej przebudowy systemu podatkowego. Główna obawa MFW koncentruje się na \textbf{elastyczności podaży pracy} (labor supply elasticity). Model przyjęty przez autorów raportu zakłada, że sfinansowanie BDP poprzez podwyżkę podatków dochodowych (PIT) lub konsumpcyjnych (VAT) doprowadzi do negatywnego szoku podażowego na rynku pracy. Zgodnie z tą logiką, gospodarstwa domowe, otrzymując transfer bezwarunkowy i płacąc wyższe podatki krańcowe, będą optymalizować swoją funkcję użyteczności poprzez substytucję pracy czasem wolnym.

Należy jednak zauważyć, że podejście MFW obarczone jest istotnym błędem poznawczym, charakterystycznym dla statycznej analizy komparatywnej. Raport implicite zakłada warunek \emph{ceteris paribus} w odniesieniu do technologii produkcji. Traktuje on strukturę kapitałową przedsiębiorstw jako zjawisko względnie stałe, a technologię jako czynnik egzogeniczny (zewnętrzny). W optyce MFW, przedsiębiorstwa reagują na BDP pasywnie -- akceptując wyższe koszty lub zmniejszając produkcję w odpowiedzi na odpływ pracowników.

Niniejsza praca stawia tezę, że w warunkach gwałtownego postępu w dziedzinie sztucznej inteligencji (AI), takie założenie jest nieuprawnione. Presja fiskalna i płacowa zidentyfikowana przez MFW nie musi prowadzić do spadku produkcji (jak w modelu neoklasycznym), lecz może stać się bodźcem do radykalnej, \textbf{endogenicznej zmiany technologii produkcji}.

\subsubsection{1.2. Mechanizm transmisji w Małej Gospodarce Otwartej: Casus Polski}
\label{mechanizm-transmisji-w-malej-gospodarce-otwartej}

Aby zrozumieć rzeczywisty wpływ BDP na decyzje menedżerskie w Polsce, należy porzucić modele budowane dla gospodarek zamkniętych (jak USA) i przenieść analizę na grunt Małej Gospodarki Otwartej z płynnym kursem walutowym i własnym bankiem centralnym.

\begin{quote}
W przeciwieństwie do Stanów Zjednoczonych, które ze względu na dominację dolara i niski udział handlu w PKB mogą być modelowane jako gospodarka relatywnie zamknięta (autonomiczna monetarnie), Polska posiada cechy klasycznej Małej Gospodarki Otwartej (SOE). Oznacza to, że krajowa polityka stóp procentowych jest ściśle skorelowana z ryzykiem kursowym, a ekspansja fiskalna (BDP) natychmiast przekłada się na presję importową i walutową, czego modele amerykańskie często nie uwzględniają.
\end{quote}

Wprowadzenie BDP w Polsce wiąże się z natychmiastowym wystąpieniem dwóch sprzężonych szoków:

\begin{enumerate}
\item \textbf{Szok popytowy:} Gwałtowny wzrost dochodu rozporządzalnego gospodarstw domowych o wysokiej krańcowej skłonności do konsumpcji.
\item \textbf{Szok importowy:} Część tego popytu zostanie skierowana na dobra importowane, co w krótkim okresie pogorszy saldo rachunku obrotów bieżących.
\end{enumerate}

W świetle teorii \textbf{Parytetu Stóp Procentowych (Interest Rate Parity)}, pogorszenie bilansu handlowego oraz presja inflacyjna wywołana kreacją pieniądza (MMT) doprowadzą do presji na osłabienie waluty krajowej (PLN). Narodowy Bank Polski, realizując cel inflacyjny i broniąc wartości złotego przed niekontrolowaną deprecjacją, będzie zmuszony do zaostrzenia polityki monetarnej.

Oznacza to, że stopa wolna od ryzyka $R_f$, a w konsekwencji stawki referencyjne WIBOR, muszą wzrosnąć powyżej poziomów notowanych w strefie Euro czy USA, aby utrzymać premię za ryzyko dla inwestorów zagranicznych.

Dla polskiego przedsiębiorcy implikacje są fundamentalne. Koszt kapitału obcego $K_d$ drastycznie rośnie. Menedżer staje w obliczu „podwójnego uderzenia'':

\begin{itemize}
\item \textbf{Wzrost kosztów operacyjnych (OPEX):} Wynikający z presji na wzrost płac (efekt płacy rezerwowej BDP).
\item \textbf{Wzrost kosztów finansowych:} Wynikający z reakcji NBP (wysokie stopy procentowe).
\end{itemize}

W takich warunkach makroekonomicznych, utrzymanie tradycyjnego modelu biznesowego opartego na pracy ludzkiej i lewarowaniu tanim długiem staje się niemożliwe.

\subsubsection{1.3. Metodologiczna luka badawcza: Błąd ergodyczności}
\label{metodologiczna-luka-badawcza-blad-ergodycznosci}

Dlaczego zatem większość modeli makroekonomicznych (w tym przywoływany model MFW oparty na logice DSGE -- \emph{Dynamic Stochastic General Equilibrium}) nie przewiduje fali bankructw lub wymuszonej automatyzacji, lecz jedynie „dostosowanie się'' gospodarki?

Źródłem tego błędu jest fundamentalne założenie o \textbf{ergodyczności} procesów gospodarczych. Modele głównego nurtu zakładają, że rynki są układami dążącymi do średniej (mean-reverting), a szoki są symetryczne. W takim ujęciu, „średnia po czasie'' (time average) dla pojedynczego podmiotu jest tożsama ze „średnią po zespole'' (ensemble average) dla całej gospodarki. Model zakłada, że firma może przetrwać okresowy wzrost kosztów, optymalizując zyski w nieskończonym horyzoncie czasowym.

W rzeczywistości menedżerskiej, jak wskazuje Ole Peters w nurtach \textbf{Ekonomii Złożoności (Complexity Economics)} \cite{peters2019,arthur2015}, rynki są układami \textbf{nieergodycznymi}. Czas i ścieżka dojścia do równowagi mają kluczowe znaczenie. Wprowadzenie BDP nie jest „drobną perturbacją'', lecz zmianą parametru kontrolnego, która może wypchnąć przedsiębiorstwo poza tzw. \textbf{barierę pochłaniającą (Absorbing Barrier)}.

Bariera pochłaniająca to stan (np. utrata płynności, bankructwo), z którego nie ma powrotu. Jeśli szok kosztowy wywołany BDP sprawi, że kapitał obrotowy firmy wyczerpie się w 3 miesiące, to fakt, iż „w długim okresie gospodarka wróci do równowagi'', jest dla menedżera nieistotny. Firma przestanie istnieć.

Stawia to menedżera przed dylematem nieliniowym. Nie chodzi o optymalizację marży o +/- 5\%. Chodzi o uniknięcie bariery pochłaniającej. Jedyną drogą ucieczki przed barierą w warunkach drogiej pracy i drogiego pieniądza jest skokowa, nieliniowa transformacja aktywów w stronę rozwiązań o zerowym koszcie krańcowym -- czyli sztucznej inteligencji.

Niniejsza praca wypełnia tę lukę badawczą, odrzucając założenie o liniowym dostosowaniu na rzecz modelu histerezy i wymuszonej substytucji technologicznej w warunkach nieergodycznych.

\setcounter{mychapter}{2}\setcounter{figure}{0}\setcounter{table}{0}
\section{ROZDZIAŁ 2}
\label{rozdzial-2}

\subsection{Teoretyczne fundamenty nowego modelu: Od równowagi do złożoności}
\label{teoretyczne-fundamenty-nowego-modelu}

Wykazawszy w \emph{Rozdziale 1} ograniczenia klasycznej analizy fiskalnej i modeli równowagi ogólnej (DSGE) w konfrontacji z szokiem BDP, niniejszy rozdział stawia sobie za cel zbudowanie alternatywnych ram teoretycznych. Nowy model, proponowany w tej pracy, opiera się na syntezie dwóch nurtów, które rzadko są łączone w literaturze przedmiotu: post-keynesowskiej Nowoczesnej Teorii Monetarnej (MMT) w ujęciu sald sektorowych oraz fizyki rynków finansowych (Ekonomii Nieergodycznej). Takie podejście pozwala na zdefiniowanie BDP nie jako kosztu budżetowego, lecz jako \textbf{impulsu zmieniającego fazę układu dynamicznego}.

\subsubsection{2.1. MMT w gospodarce otwartej: Tożsamość sald sektorowych i bariera zasobowa}
\label{mmt-w-gospodarce-otwartej}

Aby zrozumieć, dlaczego BDP w Polsce musi prowadzić do wzrostu stóp procentowych (a więc kosztu kapitału), należy odwołać się do fundamentalnej tożsamości makroekonomicznej sformułowanej przez Wynne'a Godleya \cite{godley2007}. W każdym systemie gospodarczym suma sald trzech sektorów musi wynosić zero:

\[(S - I) + (G - T) + (X - M) = 0\]

Gdzie:

\begin{itemize}
\item $(S - I)$ -- Saldo sektora prywatnego (Oszczędności minus Inwestycje).
\item $(G - T)$ -- Saldo sektora publicznego (Wydatki rządu minus Podatki).
\item $(X - M)$ -- Saldo sektora zagranicznego (Eksport minus Import).
\end{itemize}

Wprowadzenie BDP oznacza gigantyczny wzrost składnika $G$ (transfery rządowe). Jeśli, zgodnie z krytyką MFW omówioną w \emph{Rozdziale 1}, rząd nie zdecyduje się na natychmiastowe, drastyczne podniesienie podatków ($T$) -- co jest politycznie prawdopodobne -- deficyt sektora publicznego gwałtownie rośnie.

Zgodnie z logiką tożsamości księgowej, ten deficyt \textbf{musi} znaleźć swoje odzwierciedlenie w nadwyżce innego sektora. W gospodarce zamkniętej (USA) trafiłby on w całości do sektora prywatnego ($S > I$), zwiększając oszczędności gospodarstw domowych. Jednak w Małej Gospodarce Otwartej (Polska), gdzie krańcowa skłonność do importu jest wysoka (a suma obrotów handlowych przekracza 100\% PKB), znaczna część tego impulsu „wycieka'' za granicę, pogarszając saldo $(X - M)$.

Tutaj dochodzimy do kluczowego dla MMT pojęcia \textbf{bariery zasobowej (Real Resource Constraint)}. MMT nie twierdzi, że państwo może drukować pieniądze bez ograniczeń. Twierdzi, że ograniczeniem nie są finanse, lecz \textbf{realne zasoby} (praca, surowce, energia) \cite{kelton2020,mosler2010}.

W momencie wprowadzenia BDP w Polsce następuje zderzenie dwóch sił:

\begin{enumerate}
\item Nominalna ilość pieniądza rośnie (kreacja waluty przez deficyt $G - T$).
\item Realna podaż pracy maleje (ludzie wycofują się z rynku pracy dzięki BDP).
\end{enumerate}

To uderzenie w barierę zasobową w modelu MMT jest bezpośrednim wyzwalaczem inflacji. W warunkach polskich, aby zatrzymać ucieczkę kapitału (wynikającą z pogorszenia salda $X - M$), bank centralny (NBP) \textbf{musi} podnieść stopy procentowe, broniąc waluty. Tym samym, MMT dostarcza nam pierwszego filaru modelu: BDP w gospodarce otwartej nieuchronnie prowadzi do \textbf{szoku kosztu pieniądza} dla przedsiębiorstw.

\subsubsection{2.2. Nieergodyczność i bariera pochłaniająca: Czas jako zasób krytyczny}
\label{nieergodycznosc-i-bariera-pochlaniajaca}

Drugim filarem teoretycznym jest odrzucenie paradygmatu „powrotu do średniej''. W klasycznej ekonomii menedżerskiej ryzyko utożsamiane jest ze zmiennością (np. odchyleniem standardowym), a procesy rynkowe traktowane są jako \textbf{ergodyczne}.

Proces jest ergodyczny, jeśli średnia po czasie (wynik jednego podmiotu w długim okresie) jest równa średniej po zespole (średni wynik wszystkich podmiotów w danym momencie).

\[\langle x \rangle_{\text{time}} = \langle x \rangle_{\text{ensemble}}\]

W pracy opieram się na dorobku Ole Petersa \cite{peters2019,peters2018} (London Mathematical Laboratory), który udowadnia, że rynki finansowe i decyzje inwestycyjne są \textbf{nieergodyczne}. W świecie nieergodycznym, dla menedżera nie liczy się „wartość oczekiwana'' (Expected Value), lecz „dynamika wzrostu w czasie'' (Time-average Growth Rate).

Wprowadzenie BDP generuje szok kosztowy (wzrost płac i odsetek). W modelu ergodycznym (MFW), firma mogłaby ten okres „przeczekać'', notując przejściowe straty. W modelu nieergodycznym istnieje zjawisko \textbf{bariery pochłaniającej (Absorbing Barrier)} -- np. poziomu gotówki równego zero.

Jeśli w wyniku BDP koszty obsługi długu i pracy przewyższą przychody, a firma uderzy w barierę pochłaniającą (bankructwo), to fakt, że „w długim okresie AI byłoby tańsze'' lub „gospodarka się dostosuje'', jest nieistotny. Firma znika z rynku.

\textbf{Implikacja dla modelu:} Decyzja o automatyzacji nie jest w tym ujęciu optymalizacją zysku (jak w klasycznej ekonomii). Jest \textbf{strategią unikania ruiny}. Przedsiębiorstwo wdraża AI nie dlatego, że chce zwiększyć marżę o 2 punkty procentowe, ale dlatego, że tylko radykalne cięcie kosztów zmiennych (pracy) pozwala odsunąć barierę pochłaniającą w czasie i przetrwać przy drogim pieniądzu.

\subsubsection{2.3. Płaca Rezerwowa jako parametr kontrolny i bifurkacja}
\label{placa-rezerwowa-jako-parametr-kontrolny}

Ostatnim elementem układanki jest zdefiniowanie mechanizmu przejścia. Traktuję tu gospodarkę jako nieliniowy układ dynamiczny, a płacę rezerwową ($w_r$) jako jego \textbf{parametr kontrolny}.

Płaca rezerwowa to najniższa stawka, za którą jednostka decyduje się podjąć pracę. BDP podnosi ten próg skokowo. W układach nieliniowych, płynna zmiana parametru kontrolnego (wzrost kosztów pracy) może prowadzić do nagłej zmiany zachowania całego systemu -- zjawisko to nazywamy \textbf{bifurkacją}.

Do momentu wprowadzenia BDP, układ znajduje się w stabilnym stanie „Gospodarki Pracochłonnej''. Koszt pracy jest niższy niż koszt kapitału technologicznego ($w < r \cdot K$). W momencie wprowadzenia BDP, parametr kontrolny przekracza punkt krytyczny (Tipping Point). \textbf{Układ traci stabilność}.

Następuje bifurkacja ścieżek rozwoju przedsiębiorstwa:

\begin{itemize}
\item \textbf{Ścieżka A (Bierność):} Firma próbuje utrzymać zatrudnienie. Ze względu na nieergodyczność, szybko uderza w barierę pochłaniającą (koszty > przychody).
\item \textbf{Ścieżka B (Skok Technologiczny):} Firma dokonuje skokowej substytucji $L \rightarrow K_{AI}$.
\end{itemize}

Co kluczowe, w układach dynamicznych występuje zjawisko \textbf{histerezy} (zależności od ścieżki). Raz dokonana transformacja w stronę AI (inwestycja w CAPEX, zwolnienie ludzi, zmiana procesów) jest nieodwracalna. Nawet jeśli w przyszłości rząd wycofałby BDP, a płaca rezerwowa by spadła, firma nie wróci do zatrudniania ludzi, ponieważ system „zapadł się'' w nowy \textbf{atraktor} automatyzacji.

Podsumowując, ramy teoretyczne niniejszej pracy opierają się na założeniu, że BDP w Polsce wywoła nieodwracalne przejście fazowe gospodarki, wymuszone przez barierę zasobową (MMT) i ryzyko ruiny (nieergodyczność).

\setcounter{mychapter}{3}\setcounter{figure}{0}\setcounter{table}{0}
\section{ROZDZIAŁ 3}
\label{rozdzial-3}

\subsection{Model Wymuszonej Substytucji: Matematyka decyzji menedżerskiej w warunkach szoku BDP}
\label{model-wymuszonej-substytucji}

W poprzednich rozdziałach zdefiniowano środowisko makroekonomiczne (szok kosztowy w gospodarce otwartej) oraz naturę ryzyka (nieergodyczność). Niniejszy rozdział ma na celu sformalizowanie procesu decyzyjnego przedsiębiorstwa. Wykorzystując zmodyfikowaną funkcję produkcji o Stałej Elastyczności Substytucji (CES), wykażę, że w warunkach BDP następuje zerwanie relacji komplementarności między pracą a kapitałem na rzecz ich absolutnej substytucyjności. W dalszej części rozdziału przedstawię autorski model „Endogenicznej Deflacji Technologicznej'', który rozwiązuje dylemat inflacyjny MMT.

\subsubsection{3.1. Funkcja Produkcji CES i parametr $\sigma$}
\label{funkcja-produkcji-ces-i-parametr-sigma}

W klasycznej ekonomii menedżerskiej często przyjmuje się funkcję Cobba-Douglasa $Y = AK^{\alpha}L^{\beta}$, która zakłada stałą elastyczność substytucji równą 1. Oznacza to, że kapitał i praca są w pewnym stopniu wymienne, ale też komplementarne (maszyna potrzebuje operatora).

W erze BDP i generatywnej sztucznej inteligencji (GenAI), to założenie jest błędne. AI nie jest narzędziem dla pracownika; jest jego substytutem. Dlatego właściwym narzędziem analitycznym jest funkcja \textbf{CES (Constant Elasticity of Substitution)}:

\[Y = A\left[\alpha L^{\frac{\sigma - 1}{\sigma}} + (1 - \alpha)K_{AI}^{\frac{\sigma - 1}{\sigma}}\right]^{\frac{\sigma}{\sigma - 1}}\]

Gdzie:

\begin{itemize}
\item $Y$ -- Produkcja (Output).
\item $A$ -- Ogólna produktywność czynników.
\item $L$ -- Praca ludzka (koszt $w$).
\item $K_{AI}$ -- Kapitał inteligentny/AI (koszt $r$).
\item $\alpha$ -- Parametr dystrybucji (waga pracy w procesie).
\item $\sigma$ -- Elastyczność substytucji.
\end{itemize}

Klucz do modelu leży w interpretacji parametru $\sigma$.

\begin{itemize}
\item Dla tradycyjnych maszyn: $\sigma \approx 1$.
\item Dla AI (które potrafi pisać kod, obsługiwać klienta, analizować dane): $\sigma \rightarrow \infty$.
\end{itemize}

Decyzję menedżera o optymalnej relacji kapitału do pracy ($\frac{K_{AI}}{L}$) opisuje równanie optymalizacji kosztów:

\[\frac{K_{AI}}{L} = \left(\frac{1 - \alpha}{\alpha}\right)^{\sigma}\left(\frac{w}{r}\right)^{\sigma}\]

\textbf{Wnioski dla modelu BDP:}

Wprowadzenie BDP podnosi płacę rezerwową ($w \uparrow$). Bank centralny podnosi stopy procentowe ($r \uparrow$). Teoretycznie wzrost obu parametrów mógłby się znosić.

Jednakże, skoro dla AI parametr $\sigma$ jest ogromny (dąży do nieskończoności), to nawet \textbf{minimalna} zmiana relacji $\frac{w}{r}$ na niekorzyść pracy powoduje wykładniczą zmianę po lewej stronie równania.

Przedsiębiorstwo nie redukuje zatrudnienia liniowo. W momencie, gdy $w$ (koszt pracy chronionej BDP) przekracza pewien próg, stosunek $\frac{K_{AI}}{L}$ dąży do nieskończoności. Następuje całkowite wyparcie czynnika ludzkiego.

\paragraph{3.1.1. Analiza porównawcza: $\sigma$ jako amplifikator szoku BDP}
\label{analiza-porownawcza-sigma-jako-amplifikator}

Powyższa analiza jakościowa wymaga formalizacji. W szczególności należy wykazać, w jaki sposób elastyczność substytucji $\sigma$ determinuje skalę reakcji na szok kosztowy wywołany BDP. Niniejsza sekcja wyprowadza formalnie optymalny stosunek $K_{AI}/L$ z minimalizacji kosztów funkcji produkcji CES i demonstruje mechanizm amplifikacji.

\textbf{Wyprowadzenie.} Rozważmy problem minimalizacji kosztów firmy z technologią CES:

\[\min_{L,K} \, wL + rK_{AI} \quad \text{p.w.} \quad Y = A[\alpha L^{\rho} + (1-\alpha)K_{AI}^{\rho}]^{1/\rho}\]

gdzie $\rho = (\sigma-1)/\sigma$, $w$ to koszt pracy (płaca efektywna z uwzględnieniem BDP), a $r$ to user cost kapitału AI (obejmujący deprecjację, stopę procentową i utrzymanie).

\textbf{Warunki pierwszego rzędu (FOC)} z Lagrangianu dają warunek tangencji:

\[\frac{MPK_{AI}}{MPL} = \frac{r}{w}\]

Po podstawieniu pochodnych cząstkowych funkcji CES i uproszczeniu:

\[\frac{K_{AI}}{L} = \left[\frac{(1-\alpha)}{\alpha} \cdot \frac{w}{r}\right]^{\sigma}\]

\textbf{Definicja.} Niech $\theta \equiv \left[\frac{(1-\alpha)}{\alpha}\right] \cdot (w/r)$ oznacza wskaźnik presji kosztowej --- stosunek relatywnego kosztu pracy do kapitału, ważony udziałami czynników. Wówczas:

\[\frac{K_{AI}}{L} = \theta^{\sigma}\]

Ta elegancka formuła ujawnia fundamentalny mechanizm: stosunek optymalny kapitału AI do pracy jest potęgową funkcją presji kosztowej $\theta$, z wykładnikiem równym elastyczności substytucji $\sigma$.

\textbf{Trzy reżimy.} Zachowanie systemu zależy krytycznie od wartości $\theta$:

\begin{itemize}
\item $\theta < 1$ (praca relatywnie tania): $K_{AI}/L < 1$ i maleje z rosnącym $\sigma$ --- wysoka substytucyjność wzmacnia przewagę pracy, firmy unikają automatyzacji.
\item $\theta = 1$ (punkt bifurkacji): $K_{AI}/L = 1$ niezależnie od $\sigma$ --- system jest neutralny, ale niestabilny.
\item $\theta > 1$ (praca relatywnie droga): $K_{AI}/L > 1$ i rośnie wykładniczo z $\sigma$ --- wysoka substytucyjność amplifikuje automatyzację, potencjalnie bez ograniczeń.
\end{itemize}

\textbf{Analiza numeryczna z parametrami case study.} Korzystając z danych kalibracyjnych z Rozdziału 4: $\alpha = 0,70$, $w_0 = 70\,000$ PLN/pracownik/rok (pre-BDP), $w_1 = 91\,000$ PLN/pracownik/rok (post-BDP, +30\%), $r_0 = 24\,300$ PLN/jednostkę AI/rok (pre-BDP), $r_1 = 30\,000$ PLN/jednostkę AI/rok (post-BDP, stopy 4\%→12\%):

\[\theta_0 = \left[\frac{(1-0,70)}{0,70}\right] \cdot \frac{70\,000}{24\,300} = 0,4286 \cdot 2,8807 = 1,2346\]

\[\theta_1 = \left[\frac{(1-0,70)}{0,70}\right] \cdot \frac{91\,000}{30\,000} = 0,4286 \cdot 3,0333 = 1,3000\]

\[\Delta\theta = +5,3\%\]

Przesunięcie $\theta$ jest skromne --- zaledwie 5,3\%. Kluczowe jest jednak to, jak $\sigma$ amplifikuje ten niewielki szok:

\begin{itemize}
\item $\sigma = 1$ (Cobb-Douglas): $K/L$: 1,23 → 1,30 (wzmocnienie 1,1×)
\item $\sigma = 2$: $K/L$: 1,52 → 1,69 (wzmocnienie 1,1×)
\item $\sigma = 5$: $K/L$: 2,86 → 3,71 (wzmocnienie 1,3×)
\item $\sigma = 10$: $K/L$: 8,19 → 13,79 (wzmocnienie 1,7×)
\item $\sigma = 20$: $K/L$: 67,07 → 190,05 (wzmocnienie 2,8×)
\item $\sigma = 50$: $K/L$: 24\,800 → 327\,339 (wzmocnienie 13,2×)
\end{itemize}

Przy $\sigma = 50$ --- wartości realistycznej dla zastępowania rutynowych zadań kognitywnych przez LLM \cite{webb2020,acemoglu2022} --- 5-procentowa zmiana $\theta$ przekłada się na 13-krotne wzmocnienie stosunku $K_{AI}/L$. To właśnie ten mechanizm amplifikacji wyjaśnia, dlaczego pozornie umiarkowany szok BDP może wywołać lawinową automatyzację.

\textbf{Paradoks $\theta > 1$ i pozorna racjonalność pre-BDP.} Kluczowa obserwacja: $\theta_0 = 1,23 > 1$ już przed wprowadzeniem BDP. Z perspektywy statycznej optymalizacji CES, automatyzacja jest racjonalna nawet bez szoku. Dlaczego więc firmy nie automatyzują?

Odpowiedź leży w nieergodyczności procesu decyzyjnego \cite{peters2019}. Model CES zakłada średnią zespołową --- optymalizację nad kontinuum stanów natury. Realna firma podejmuje jednorazową, nieodwracalną decyzję inwestycyjną w warunkach niepewności. Bariera pochłaniająca (bankructwo przy nieudanej implementacji AI) tworzy efektywny próg $\theta^* > \theta = 1$, powyżej którego oczekiwana wartość inwestycji staje się dodatnia dopiero po uwzględnieniu ryzyka ruiny.

BDP działa zatem podwójnie: (1) przesuwa $\theta$ w górę (z 1,23 do 1,30), zwiększając zachętę do automatyzacji, oraz (2) obniża efektywny próg $\theta^*$, ponieważ firmy wiedzą, że szok kosztowy jest permanentny i dotyczy całego rynku --- co redukuje niepewność co do relatywnej opłacalności. Oba efekty kumulują się, a $\sigma$ amplifikuje ich łączny wpływ potęgowo.

\begin{figure}[H]
\centering
\includegraphics[width=0.9\textwidth]{image1.png}
\caption{Ewolucja izokwanty produkcji: Od komplementarności pracy i kapitału (model tradycyjny) do pełnej substytucyjności (model AI). Panel lewy pokazuje, jak $\sigma$ wyostrza przejście z dominacji pracy ($\theta < 1$) do dominacji AI ($\theta > 1$). Panel prawy pokazuje rozbieżność trajektorii pre- i post-BDP jako funkcję $\sigma$ --- przy wysokiej substytucyjności nawet minimalny szok $\Delta\theta$ generuje eksplozywną różnicę w optymalnym $K_{AI}/L$.}
\label{fig:ces-evolution}
\end{figure}

\subsubsection{3.2. Paradoks Zerowego Kosztu Krańcowego}
\label{paradoks-zerowego-kosztu-krancowego}

Krytycy automatyzacji przy BDP często podnoszą argument: \emph{„Skoro stopy procentowe ($r$) będą wysokie, to firm nie będzie stać na inwestycje w drogie serwery i AI''}.

Jest to myślenie oparte na błędnej strukturze kosztów. Należy rozróżnić strukturę kosztów pracy i AI:

\begin{enumerate}
\item \textbf{Praca ($L$):} Niskie nakłady początkowe (CAPEX), ale wysoki i rosnący koszt krańcowy (OPEX). Każda kolejna jednostka produktu wymaga kolejnej godziny pracy.

\[TC_L(Y) = w \cdot L(Y)\]

\item \textbf{AI ($K_{AI}$):} Gigantyczne nakłady początkowe (CAPEX -- zakup modelu, wdrożenie), ale koszt krańcowy bliski zera.

\[TC_{AI}(Y) = FC(r) + \epsilon \cdot Y\]

Gdzie $\epsilon \rightarrow 0$.
\end{enumerate}

W warunkach BDP (szok popytowy), popyt na dobra masowe ($Y$) rośnie. Obliczmy Koszt Przeciętny ($AC$) dla obu wariantów:

\begin{itemize}
\item Dla Pracy: $\lim_{Y \rightarrow \infty} AC_L = w$ (koszt jest sztywny, wyznaczony przez BDP).
\item Dla AI: $\lim_{Y \rightarrow \infty} AC_{AI} = \frac{FC(r)}{Y} = 0$.
\end{itemize}

\textbf{Wniosek:} Nawet przy bardzo wysokim koszcie pieniądza ($r$), który podbija koszty stałe ($FC$), korzyści skali wynikające z zerowego kosztu krańcowego AI sprawiają, że przy masowej produkcji technologia ta jest bezkonkurencyjna. BDP, stymulując popyt masowy, faworyzuje technologie o zerowym $MC$, a karze technologie oparte na pracy ($AC = w$).

\subsubsection{3.3. Endogeniczna Deflacja Technologiczna w modelu MMT}
\label{endogeniczna-deflacja-technologiczna}

W tym punkcie dochodzimy do syntezy makroekonomicznej, która stanowi główny wkład innowacyjny tej pracy. Rozwiązujemy tu dylemat inflacyjny zidentyfikowany przez MFW.

Klasyczne równanie wymiany (w ujęciu dynamicznym) to:

\[\pi = \Delta M + \Delta V - \Delta Y\]

Gdzie $\pi$ to inflacja.

W standardowym modelu MMT (bez AI), wprowadzenie BDP powoduje wzrost podaży pieniądza ($\Delta M > 0$) i prędkości obiegu ($\Delta V > 0$). Jeśli podaż dóbr ($\Delta Y$) jest ograniczona przez brak rąk do pracy (spadek podaży pracy), inflacja $\pi$ wybucha.

\textbf{Zmieniona Funkcja Produkcji:}

Wprowadzamy do równania wymuszoną automatyzację. Skoro firmy, uciekając przed bankructwem (\emph{Rozdział 2}), masowo inwestują w AI (\emph{Rozdział 3.1}), to potencjał produkcyjny gospodarki ($\Delta Y$) przestaje zależeć od demografii, a zaczyna zależeć od mocy obliczeniowej.

Zgodnie z Prawem Moore'a (i jego odpowiednikami dla AI), efektywność kapitału $K_{AI}$ rośnie wykładniczo. Otrzymujemy zatem:

\[\Delta Y_{AI} \gg \Delta M_{BDP}\]

Wzrost produktywności wywołany przez AI jest szybszy niż tempo kreacji pieniądza przez rząd na potrzeby BDP.

Formalnie, warunkiem wystarczającym deflacji technologicznej jest: $g_Y > g_M + g_V$, gdzie $g_Y$ = tempo wzrostu realnej produkcji (napędzane przez AI), $g_M$ = tempo wzrostu podaży pieniądza (deficyt na BDP), a $g_V$ = tempo zmian prędkości obiegu. Przy $g_Y$ rosnącym wykładniczo (prawo Moore'a / scaling laws dla LLM \cite{aghion2018,nordhaus2021}) i $M$ rosnącą liniowo (stały transfer BDP), a zatem $g_M$ malejącym w czasie, nierówność jest spełniona po przekroczeniu masy krytycznej automatyzacji --- co symulacja SFC-ABM identyfikuje w okolicach miesiąca 95.

Prowadzi to do zjawiska, które definiuję jako \textbf{„Endogeniczna Deflacja Technologiczna''}.

W tym modelu AI pełni rolę \textbf{„gąbki inflacyjnej'' (Inflationary Absorption Buffer)}. Zamiast ściągać nadmiar pieniądza podatkami (co postuluje MFW i klasyczne MMT), gospodarka „wchłania'' ten pieniądz poprzez dostarczenie na rynek gigantycznej ilości tanich usług i dóbr cyfrowych wytworzonych przez automaty.

\textbf{Konkluzja Rozdziału:}

Wysoki koszt pieniądza i wysoka płaca rezerwowa nie są przeszkodami dla gospodarki. Są \textbf{niezbędnymi bodźcami}, które wymuszają transformację technologiczną. Bez presji kosztowej BDP, firmy nie zainwestowałyby w AI wystarczająco szybko, by zrównoważyć bilans monetarny państwa. Paradoksalnie, to „drożyzna'' ratuje wartość pieniądza poprzez wymuszenie efektywności.

\setcounter{mychapter}{4}\setcounter{figure}{0}\setcounter{table}{0}
\section{ROZDZIAŁ 4}
\label{rozdzial-4}

\subsection{Symulacja Bilansowa i Studium Przypadku: Mechanizm Transmisji w Polskiej Gospodarce Otwartej}
\label{symulacja-bilansowa-i-studium-przypadku}

Wykazawszy w poprzednich rozdziałach teoretyczną nieuchronność procesu automatyzacji w warunkach szoku BDP (poprzez model funkcji produkcji CES oraz barierę nieergodyczności), niniejszy rozdział ma na celu operacjonalizację tych pojęć. W pierwszej części zaprezentowano makroekonomiczny przepływ funduszy przy użyciu T-kont, ilustrujący transmisję impulsu fiskalnego do sektora bankowego i przedsiębiorstw. W drugiej części przeprowadzono szczegółową symulację finansową (Case Study) dla hipotetycznego przedsiębiorstwa usługowego, wykazując matematycznie, dlaczego wysoki koszt pieniądza paradoksalnie przyspiesza decyzje inwestycyjne w technologie o zerowym koszcie krańcowym.

\subsubsection{4.1. Makroekonomia T-kont: Od deficytu budżetowego do długu korporacyjnego}
\label{makroekonomia-t-kont}

Zgodnie z paradygmatem Nowoczesnej Teorii Monetarnej (MMT), pieniądz wprowadzany do gospodarki poprzez deficyt budżetowy finansujący BDP nie znika, lecz musi znaleźć odzwierciedlenie w bilansach sektora prywatnego. W gospodarce otwartej (Polska), kluczowe jest prześledzenie drogi tego pieniądza, aby zrozumieć presję na stopy procentowe.

Poniższa analiza T-kont (Sald Sektorowych) ilustruje sekwencję zdarzeń po wprowadzeniu BDP.

\textbf{Etap 1: Kreacja Pieniądza (Impuls Fiskalny)}

Skarb Państwa emituje obligacje (lub korzysta z mechanizmów bezpośrednich), a NBP kredytuje system bankowy. Banki komercyjne księgują nowe depozyty na rachunkach obywateli (transfer BDP).

\textbf{T-Konta w momencie T0 (Wprowadzenie BDP):}

\begin{itemize}
\item \textbf{Sektor Gospodarstw Domowych:}
\begin{itemize}
\item $\uparrow$ Aktywa: Depozyty bieżące (Środki z BDP).
\item $\uparrow$ Pasywa: Kapitał własny (Wzrost dochodu rozporządzalnego).
\end{itemize}
\item \textbf{Sektor Bankowy:}
\begin{itemize}
\item $\uparrow$ Pasywa: Depozyty klientów (Nowy pieniądz).
\item $\uparrow$ Aktywa: Rezerwy w NBP (Płynność nadwyżkowa).
\end{itemize}
\end{itemize}

\textbf{Etap 2: Reakcja Monetarna (Szok Kosztowy)}

Wzrost masy pieniężnej ($M2$) i popytu konsumpcyjnego w Polsce wywołuje presję inflacyjną oraz presję na import (pogorszenie salda obrotów bieżących). Aby bronić kursu złotego (PLN) i ściągnąć nadmiar płynności, Rada Polityki Pieniężnej (RPP) drastycznie podnosi stopy procentowe.

\begin{itemize}
\item \textbf{Efekt:} Koszt pieniądza na rynku międzybankowym (WIBOR) rośnie skokowo (np. z 5\% do 12\%).
\end{itemize}

\textbf{Etap 3: Konwersja w Przedsiębiorstwach (Moment Inwestycyjny)}

To kluczowy moment transmisji. Firmy otrzymują zwiększone przychody (ludzie wydają BDP), ale napotykają barierę kosztową pracy. Aby utrzymać rentowność, dokonują konwersji bilansowej: zaciągają kredyt inwestycyjny (kreacja nowego pieniądza przez banki pod zastaw aktywów) na zakup technologii.

\subsubsection{4.2. Studium Przypadku}
\label{studium-przypadku}

Aby nadać analizie wymiar menedżerski, posłużmy się modelem finansowym dla typowej polskiej firmy z sektora BPO/SSC (Business Process Outsourcing), który jest silnie eksponowany na ryzyko automatyzacji przez Generatywną AI.

\textbf{Założenia wyjściowe (Status Quo -- Przed BDP):}

Firma działa w modelu praco-chłonnym, typowym dla polskiej gospodarki przed transformacją.

\begin{itemize}
\item \textbf{Przychody ze sprzedaży:} 100 mln PLN.
\item \textbf{Zatrudnienie:} 1000 FTE (Full Time Equivalent).
\item \textbf{Średni koszt pracodawcy (Wynagrodzenie + ZUS):} 70 000 PLN rocznie.
\item \textbf{Całkowite Koszty Pracy (OPEX):} 70 mln PLN (70\% przychodów).
\item \textbf{Pozostałe koszty stałe:} 20 mln PLN.
\item \textbf{EBITDA (Zysk operacyjny przed amortyzacją):} 10 mln PLN (Marża 10\%).
\item \textbf{Koszt długu (WIBOR + Marża):} 4\%.
\end{itemize}

\subsubsection{4.3. Scenariusz Szoku BDP: Parametry wejściowe}
\label{scenariusz-szoku-bdp-parametry-wejsciowe}

W momencie $T_1$ następuje wprowadzenie BDP. Model przyjmuje następujące zmienne egzogeniczne:

\begin{enumerate}
\item \textbf{Szok Popytowy:} Nominalne przychody firmy rosną o \textbf{10\%} (do 110 mln PLN) dzięki zwiększonej sile nabywczej konsumentów.
\item \textbf{Szok Płacowy (Płaca Rezerwowa):} Aby zatrzymać pracownika, który otrzymuje BDP, firma musi podnieść wynagrodzenia o \textbf{30\%}.
\item \textbf{Szok Odsetkowy:} W reakcji na inflację, NBP podnosi stopy. Całkowity koszt obsługi długu firmy rośnie z 4\% do \textbf{12\%}.
\end{enumerate}

Przedsiębiorstwo staje przed punktem bifurkacji. Przeanalizujmy dwa scenariusze w oparciu o rachunek zysków i strat (P\&L).

\subsubsection{4.4. Analiza Porównawcza Scenariuszy}
\label{analiza-porownawcza-scenariuszy}

\textbf{Ścieżka A: Bierność}

Zarząd decyduje się na utrzymanie zatrudnienia, akceptując wyższe koszty pracy, licząc na to, że wzrost przychodów pokryje straty.

\textbf{Symulacja P\&L:}

\begin{itemize}
\item Przychody: 110 mln PLN.
\item Koszty Pracy: 1000 osób × (70 tys. × 1,30) = 91 mln PLN
\item Pozostałe koszty: 20 mln PLN.
\item \textbf{EBITDA:} 110 -- 91 -- 20 = -1 mln PLN
\end{itemize}

\textbf{Diagnoza:} Firma traci rentowność operacyjną. Przy ujemnej EBITDA traci zdolność kredytową (kowenanty bankowe zostają złamane). W warunkach nieergodycznych (bankructwo jest nieodwracalne), firma uderza w \textbf{barierę pochłaniającą} (bankructwo) w ciągu 6-12 miesięcy.

\emph{Wniosek:} Model oparty na pracy ludzkiej w warunkach BDP jest matematycznie niemożliwy do utrzymania.

\textbf{Ścieżka B: Wymuszona Substytucja (Ucieczka do przodu)}

Zarząd podejmuje decyzję o radykalnej automatyzacji procesów (zastąpienie Call Center botami AI).

\begin{itemize}
\item \textbf{Restrukturyzacja:} Zwolnienie 70\% załogi (700 osób). Pozostaje 300 kluczowych specjalistów.
\item \textbf{CAPEX (Inwestycja w AI):} Firma zaciąga kredyt inwestycyjny w wysokości \textbf{50 mln PLN} na zakup infrastruktury i licencji AI.
\item \textbf{Koszt finansowania:} Bardzo wysoki (12\%), co daje \textbf{6 mln PLN} odsetek rocznie.
\end{itemize}

\textbf{Symulacja P\&L:}

\begin{enumerate}
\item Przychody: 110 mln PLN (AI utrzymuje wolumen obsługi 24/7).
\item Nowe Koszty Pracy: 300 osób × 91 tys. = 27,3 mln PLN
\item Koszty utrzymania AI (Chmura/Licencje): 5 mln PLN.
\item Pozostałe koszty: 20 mln PLN.
\item \textbf{EBITDA:} 110 - 27,3 - 5 - 20 = 57,7 mln PLN
\item Odsetki (12\% od 50 mln): 6 mln PLN. Amortyzacja (CAPEX 50 mln / 10 lat): 5 mln PLN.
\item \textbf{Zysk Brutto (EBT): 57,7 - 6 - 5 = 46,7 mln PLN.}
\end{enumerate}

\subsubsection{4.5. Interpretacja Wyników: Paradoks Drogiego Pieniądza}
\label{interpretacja-wynikow-paradoks-drogiego-pieniadza}

Przeprowadzona symulacja ujawnia fundamentalny paradoks, który podważa klasyczne intuicje na temat stóp procentowych.

Zazwyczaj uważa się, że wysokie stopy procentowe (12\%) hamują inwestycje (CAPEX). Jednak w analizowanym przypadku, kluczowa jest \textbf{relatywna zmiana kosztów}.

\begin{itemize}
\item Wzrost kosztów pracy (o 21 mln PLN w Ścieżce A) jest wielokrotnie bardziej dotkliwy niż koszt obsługi długu na AI (6 mln PLN w Ścieżce B), nawet przy "lichwiarskim" oprocentowaniu.
\end{itemize}

Działa tu mechanizm \textbf{dźwigni operacyjnej}. AI, będąc kosztem stałym (CAPEX), pozwala wyeliminować gigantyczny, napompowany przez BDP koszt zmienny (Pracę).

\textbf{Wnioski dla modelu:}

\begin{enumerate}
\item \textbf{Filtr Ewolucyjny:} BDP działa jak sito. Firmy, które nie mają zdolności kredytowej lub technicznej, by wdrożyć AI (Ścieżka A), zostaną wyeliminowane z rynku.
\item \textbf{Akceleracja:} Wysoki koszt pieniądza nie zatrzymał inwestycji w AI. Przeciwnie -- uczynił ją jedyną opcją ratunkową. Gdyby stopy były niskie, a płaca rezerwowa niska, firma mogłaby zwlekać z automatyzacją. Przy "podwójnym szoku", automatyzacja stała się warunkiem \emph{sine qua non} przetrwania.
\end{enumerate}

Tym samym, symulacja potwierdza tezę pracy: w gospodarce otwartej z własną walutą, mechanizm rynkowy (stopy procentowe + płace) skutecznie wymusza transformację technologiczną, której nie wymusiłyby same dotacje czy strategie rządowe.

\subsubsection{4.6. Walidacja Empiryczna: Dowody z Gospodarki Realnej}
\label{walidacja-empiryczna-dowody-z-gospodarki-realnej}

Przeprowadzona symulacja agentowa generuje predykcje jakościowe, które można skonfrontować z dostępnymi danymi empirycznymi. Niniejsza sekcja zestawia dowody z trzech obszarów: (1) wpływu podwyżek płac na automatyzację \cite{lordan2018,hao2023,dauth2025}, (2) eksperymentów z BDP \cite{standing2017}, oraz (3) globalnych trendów pogłębiania kapitału (capital deepening) \cite{acemoglu2022}.

\paragraph{4.6.1. Szok kosztowy a inwestycje w automatyzację}
\label{szok-kosztowy-a-inwestycje-w-automatyzacje}

Centralną predykcją modelu jest to, że wzrost kosztów pracy wymusza inwestycje w technologię. Dowody empiryczne potwierdzają tę zależność z zaskakującą siłą statystyczną.

\textbf{Lordan i Neumark (2018)} \cite{lordan2018} w analizie danych CPS z lat 1980--2015 wykazali, że 10-procentowy wzrost płacy minimalnej w USA redukuje udział automatyzowalnych miejsc pracy wśród pracowników niskokwalifikowanych z elastycznością ‒0,10, przy czym w sektorze wytwórczym elastyczność ta sięga ‒0,18. Efekt koncentruje się na starszych pracownikach, kobietach i mniejszościach etnicznych -- co odpowiada asymetrycznej dystrybucji kosztów transformacji w modelu (firmy o niskiej gotowości cyfrowej).

\textbf{Hao, Li i Phaneuf (2023)} \cite{hao2023} na danych z pojedynczych zakładów pracy wykazali, że przedsiębiorstwa eksponowane na stanowe podwyżki płacy minimalnej zwiększały budżety IT o 10 328--66 808 USD rocznie w ciągu trzech lat od podwyżki. Jest to dokładnie mechanizm transmisji opisany w Rozdziale 3: szok kosztowy po stronie pracy ($w$ rośnie) przekłada się na przesunięcie optymalnego stosunku $K_{AI}/L$ w funkcji produkcji CES.

\textbf{Dauth et al. (2025)} \cite{dauth2025} potwierdził kierunkową specyficzność tego efektu: 1-procentowy wzrost płac niskokwalifikowanych prowadzi do 2--5-procentowego wzrostu innowacji automatyzacyjnych, podczas gdy innowacje nieautomatyzacyjne (np. efektywność energetyczna) nie reagują na szok płacowy. Co kluczowe, niemieckie reformy Hartza, które obniżyły płace, spowodowały spadek innowacji automatyzacyjnych -- potwierdzając odwrotny kierunek przyczynowości i walidując tezę o BDP jako katalizatorze.

\paragraph{4.6.2. Naturalne eksperymenty: McDonald's i Amazon}
\label{naturalne-eksperymenty-mcdonalds-i-amazon}

Najbardziej widocznym case study transformacji wymuszonej szokiem kosztowym jest reakcja sektora fast-food na ruch „Fight for \$15'' w USA.

\textbf{McDonald's} zakończył wdrażanie kiosków samoobsługowych we wszystkich restauracjach do 2020 r., bezpośrednio po fali legislacyjnych podwyżek płacy minimalnej. W 2024 r. wprowadzono kioski akceptujące gotówkę, eliminując ostatnią potrzebę interakcji z kasjerem. Po wprowadzeniu przez stan Kalifornia płacy minimalnej 20 USD/h dla sektora gastronomicznego (kwiecień 2024), jeden z franczyzobiorców publicznie przyznał, że „dramatycznie skrócił'' 5--10-letni plan wdrożenia automatyzacji do natychmiastowej realizacji.

To dokładnie odpowiada dynamice symulacji: po szoku firmy o wysokiej gotowości cyfrowej natychmiast automatyzują, a te o niskiej -- bankrutują lub desperacko redukują etaty.

\textbf{Amazon} dostarcza drugiego empirycznego potwierdzenia. Flota robotów magazynowych wzrosła z ok. 200 000 (2020) do ponad 1 000 000 (2025). Liczba paczek na pracownika wzrosła z ok. 175 (2016) do ok. 3 870 (2025) -- 22-krotny wzrost produktywności. Zautomatyzowany magazyn zatrudnia ok. 25\% mniej ludzi niż tradycyjny.

\paragraph{4.6.3. Programy pilotażowe BDP: Efekt płacy rezerwowej}
\label{programy-pilotazowe-bdp-efekt-placy-rezerwowej}

Kluczowym kanałem transmisji w modelu jest wzrost płacy rezerwowej ($w$). Dane z eksperymentów BDP potwierdzają ten mechanizm.

\textbf{Fiński eksperyment (2017--2018)} objął 2 000 osób otrzymujących 560 EUR miesięcznie. W pierwszym roku efekt zatrudnieniowy był pomijalny (+0,5 dnia pracy). W drugim roku beneficjenci pracowali o 6 dni więcej niż grupa kontrolna (+8,3\%). Minimalny krótkoterminowy wpływ na podaż pracy jest kluczowy: BDP podnosi reservation wage -- osoby nie podejmują dowolnej pracy, lecz szukają lepszych warunków. Z perspektywy pracodawcy oznacza to wzrost efektywnych kosztów pozyskania pracownika.

\textbf{GiveDirectly w Kenii (2017--)} w programie obejmującym 23 000 osób w 195 wioskach wykazał, że beneficjenci BDP nie redukowali podaży pracy, lecz przesuwali się z pracy najemnej do samozatrudnienia. Jest to dokładnie mechanizm, który w skali makro pozbawia firmy dostępu do taniej pracy najemnej.

\paragraph{4.6.4. Capital Deepening: Dane globalne}
\label{capital-deepening-dane-globalne}

Model predykuje, że gospodarki z wyższymi kosztami pracy powinny wykazywać wyższy stosunek kapitału do pracy ($K/L$). Dane International Federation of Robotics (IFR 2024) potwierdzają tę zależność:

\begin{itemize}
\item Korea Płd.: 1 012 robotów na 10 000 pracowników (najniższa dzietność na świecie, najwyższe płace w Azji).
\item Singapur: 770 robotów na 10 000 pracowników (wysoki koszt pracy, brak zaplecza demograficznego).
\item Chiny: 470 robotów na 10 000 pracowników (gwałtownie rosnące płace -- podwojenie w dekadę).
\item Japonia: 419 robotów na 10 000 pracowników (starzejąca się populacja, kurczący się rynek pracy).
\item Niemcy: ok. 429 robotów na 10 000 pracowników (wysoka płaca, silne związki zawodowe).
\item USA: 295 robotów na 10 000 pracowników (umiarkowane koszty, ale wzrostowy trend).
\item Polska: ok. 62 roboty na 10 000 pracowników (niskie koszty pracy, niski poziom automatyzacji).
\end{itemize}

Globalna gęstość robotów podwoiła się z 74 do 162 na 10 000 pracowników w ciągu zaledwie siedmiu lat (2016--2023). Acemoglu i Restrepo (2022) \cite{acemoglu2022} wykazali, że samo starzenie się populacji (i wynikający z niego wzrost kosztów pracy) odpowiada za niemal połowę międzynarodowej wariancji w adopcji robotów.

Polska, z gęstością robotów ok. 62 na 10 000 pracowników, zajmuje pozycję daleko poniżej średniej OECD. Wprowadzenie BDP stanowiłoby szok analogiczny do dekad wzrostu płac w Korei Płd. -- skompresowany do kilku lat. Model predykuje, że efektem byłby gwałtowny wzrost stosunku $K/L$, z towarzyszącą falą bankructw firm nieprzygotowanych cyfrowo.

\paragraph{4.6.5. Podsumowanie: Łańcuch przyczynowy walidowany empirycznie}
\label{podsumowanie-lancuch-przyczynowy-walidowany-empirycznie}

Zebrane dowody układają się w spójny łańcuch transmisji, odpowiadający mechanizmowi opisanemu w Rozdziałach 2--4:

\begin{enumerate}
\item BDP podnosi płacę rezerwową (Finlandia: minimalny krótkoterminowy efekt zatrudnieniowy; Kenia: przejście do samozatrudnienia).
\item Firmy stają wobec wyższych efektywnych kosztów pracy (analogicznie do podwyżek płacy minimalnej).
\item Firmy zwiększają inwestycje technologiczne (Hao et al.: +10 328--66 808 USD IT/zakład/rok).
\item Rośnie innowacyjność automatyzacyjna (UZH: +1\% płac → +2--5\% innowacji automatyzacyjnych).
\item Gęstość robotów rośnie, stosunek $K/L$ rośnie (IFR: podwojenie gęstości globalnej w 7 lat).
\item Udział pracy w dochodzie spada (Autor i Salomons, 2018 \cite{autor2018}: efekt narastający w czasie).
\end{enumerate}

Ten łańcuch stanowi empiryczną walidację modelu „Paradoksu Akceleracji''. BDP nie jest programem pasywnej redystrybucji -- jest endogenicznym mechanizmem wymuszającym transformację technologiczną, działającym przez kanał kosztu pracy w warunkach nieergodycznych rynków.

\subsubsection{4.7. Symulacja agentowa SFC-ABM: Dynamika
systemowa}\label{symulacja-agentowa-sfc-abm-dynamika-systemowa}

Przedstawiona w sekcjach 4.2--4.4 analiza statyczna porównuje dwa
scenariusze (Ścieżka A i B) w punkcie równowagi. Nie uchwytuje jednak
dynamiki przejścia: tempa adopcji, sprzężeń zwrotnych między sektorami,
ani emergentnych efektów interakcji tysięcy heterogenicznych firm z
systemem finansowym i instytucjonalnym. W tym celu przeprowadzono
symulację agentową w paradygmacie SFC-ABM (Stock-Flow Consistent
Agent-Based Model), w duchu prac Godleya i Lavoie \cite{godley2007}. Model obejmuje
10 000 firm w 4 sektorach (BPO/SSC, przemysł, handel/usługi, ochrona
zdrowia/budownictwo), połączonych siecią Wattsa-Strogatz (small-world, k
= 6, p = 0,10). Każdy scenariusz uruchomiono 100 razy (Monte Carlo) z
różnym ziarnem generatora, uzyskując rozkłady wyników zamiast
pojedynczych trajektorii.

\textbf{Architektura modelu.} Oprócz 4 sektorów gospodarczych, model obejmuje 6 bloków bilansowych (w sensie rachunkowości SFC): (1)
10 000 firm-agentów z heterogenicznymi parametrami (gotowość cyfrowa, profil ryzyka, mnożnik kosztu innowacji), podejmujących endogeniczne decyzje
technologiczne (Traditional → Hybrid → Automated | Bankrupt);
(2) zagregowany sektor gospodarstw domowych z logistyczną krzywą podaży
pracy; (3) rząd finansujący BDP i wydatki bazowe z CIT i VAT; (4) system
bankowy z kreacją kredytu, NPL i wymogiem CAR (Bazylea III); (5) NBP z
endogeniczną regułą Taylora; (6) sektor zagraniczny z kursem walutowym
determinowanym bilansem płatniczym i parytetem stóp procentowych (IRP).

\textbf{Kalibracja.} Parametry bazowe odpowiadają case study z sekcji
4.2: płaca bazowa 5 833 PLN/mies. ($\approx$ 70 000 PLN/rok), CAPEX AI =
1 200 000 PLN, CAPEX hybrydy = 350 000 PLN, stopa referencyjna 5,5\%,
kurs 4,50 PLN/EUR. BDP = 2 000 PLN/os./mies. aktywowane w miesiącu 30
(po 2,5 roku stabilnej gospodarki). Symulacja obejmuje 120 miesięcy (10
lat).

\textbf{Non-ergodicity discount.} Kluczowym elementem kalibracji
jest premia za niepewność (= 0,15) w fazie pre-BDP,
modulowana lokalnym efektem demonstracji w sieci Wattsa-Strogatz: jeśli
ponad 40\% sąsiadów firmy zaadoptowało AI, postrzegana niepewność spada
(efekt demonstracji). Choć wskaźnik presji kosztowej $\theta$ >
1 wskazuje na racjonalność automatyzacji (por. sekcja 3.1.1), firmy nie
adoptują AI ze względu na ryzyko ruiny przy nieudanej implementacji ---
zgodnie z nieergodycznym modelem decyzyjnym Petersa \cite{peters2019}. Dopiero szok
BDP, jako permanentny i rynkowy, znosi tę premię.

Rysunek~\ref{fig:inflation-dynamics} przedstawia dynamikę inflacji i reakcję
polityki monetarnej NBP. W fazie pre-BDP (miesiące 1--29) inflacja
oscyluje wokół 1\%, a stopa referencyjna konwerguje z inicjalnych 5,5\%
do $\sim$4\% (neutralna stopa reguły Taylora). Po wprowadzeniu
BDP w miesiącu 30 inflacja rośnie skokowo do 13,7\% (szczyt w M40), na
co NBP reaguje agresywnym zacieśnieniem polityki monetarnej --- stopa
referencyjna osiąga 20\%, a stopa kredytowa 27\%. Reguła Taylora okazuje
się skuteczna: inflacja stopniowo spada, osiągając cel 2,5\% w okolicach
M95, a w ostatniej fazie symulacji przechodzi w lekką deflację
technologiczną (−0,6\% w M120) --- potwierdzając mechanizm opisany w
sekcji 3.3.

\begin{figure}[H]
\centering
\includegraphics[width=0.95\textwidth]{image4.png}
\caption{Dynamika inflacji i reakcja polityki pieniężnej NBP.}
\label{fig:inflation-dynamics}
\end{figure}

Rysunek~\ref{fig:foreign-sector} ilustruje reakcję sektora zagranicznego. Kurs
PLN/EUR deprecjonuje z 4,50 do 4,78 (szczyt w M80), co odpowiada ok.
6-procentowemu osłabieniu złotego. Deprecjacja wynika z (1) pogorszenia
bilansu handlowego (wzrost importu konsumpcyjnego finansowanego BDP)
oraz (2) importu technologii AI. Bilans handlowy pogarsza się gwałtownie
po szoku (−80 mln PLN/mies.), lecz stopniowo odbudowuje się dzięki
rosnącej produktywności firm hybrydowych i zautomatyzowanych. W końcowej
fazie kurs stabilizuje się i lekko umacnia, gdy wyższe stopy procentowe
przyciągają kapitał zagraniczny (IRP).

\begin{figure}[H]
\centering
\includegraphics[width=0.95\textwidth]{image5.png}
\caption{Reakcja sektora zagranicznego: kurs PLN/EUR i bilans handlowy.}
\label{fig:foreign-sector}
\end{figure}

Rysunek~\ref{fig:wages-unemployment} ukazuje fundamentalny paradoks modelu: krzywe
płac i bezrobocia przecinają się, tworząc „nożyce automatyzacji".
Bezpośrednio po szoku BDP płaca rynkowa rośnie (z 6 060 do 6 470 PLN,
peak M40) ze względu na zwiększoną płacę rezerwową i napięty rynek
pracy. Jednak w miarę jak firmy przechodzą na model hybrydowy (redukując
zatrudnienie o połowę), popyt na pracę systematycznie spada. Bezrobocie
rośnie z naturalnych 1,7\% do 31\% w M120. Płaca rynkowa spada poniżej
poziomu pre-BDP (5 502 vs. 5 833), lecz BDP (2 000 PLN/os.) zapewnia
minimum egzystencjalne. To jest sedno paradoksu akceleracji: instrument
mający chronić pracowników katalizuje proces, który strukturalnie
redukuje popyt na ich pracę. Należy podkreślić, że raportowane 31\%
obejmuje zarówno klasyczne bezrobocie (brak popytu na pracę), jak i
dobrowolną nieaktywność zawodową (voluntary non-participation) ---
osoby, dla których płaca rynkowa (5~502 PLN) jest niewystarczająco
atrakcyjna wobec BDP (2~000 PLN) i płacy rezerwowej (5~000 PLN). W
pełnym modelu rynku pracy ta druga kategoria stanowi większość, co
zmienia interpretację polityczną: problem nie polega na braku miejsc
pracy, lecz na nieatrakcyjności oferowanych warunków.

\begin{figure}[H]
\centering
\includegraphics[width=0.95\textwidth]{image6.png}
\caption{Płace i bezrobocie --- nożyce automatyzacji.}
\label{fig:wages-unemployment}
\end{figure}

Rysunek~\ref{fig:tech-structure} dokumentuje ewolucję struktury technologicznej
10 000 firm. W fazie pre-BDP adopcja jest minimalna (uncertainty
discount). Po szoku obserwujemy klasyczną S-krzywą dyfuzji: model
hybrydowy (pomarańczowy) rośnie z 3\% w M30 do 54\% w M120, podczas gdy
pełna automatyzacja (zielony) stabilizuje się na poziomie 4\%. Dominacja
hybryd nad pełną automatyzacją wynika z niższego CAPEX (350 vs. 1 200
tys. PLN) i niższego progu gotowości cyfrowej ($\geq$ 0,20 vs.
0,35). Bankruci (szary, górna krawędź) stanowią 1,6\% populacji ---
głównie firmy o niskiej gotowości cyfrowej, które nie przetrwały szoku
kosztowego ani nie zdołały się zmodernizować.

\begin{figure}[H]
\centering
\includegraphics[width=0.95\textwidth]{image7.png}
\caption{Ewolucja struktury technologicznej 10\,000 firm.}
\label{fig:tech-structure}
\end{figure}

Rysunek~\ref{fig:npl} pokazuje wskaźnik kredytów zagrożonych (NPL) w
systemie bankowym. NPL ratio oscyluje między 0,2\% a 1,3\% --- daleko
poniżej progu ostrzegawczego 5\%. Oznacza to, że system bankowy
absorbuje falę bankructw bez kryzysu systemowego. Kluczowy jest tu
mechanizm: firmy bankrutują z relatywnie niskim zadłużeniem (porażki
implementacji na wczesnym etapie), a wysoka stopa odzysku (30\%) i
ciągły dochód odsetkowy z rosnącego portfela kredytowego utrzymują
adekwatność kapitałową banków. Pętla zwrotna ⑤ (bankructwa → NPL →
credit crunch → więcej bankructw) nie uruchamia się w pełni.

\begin{figure}[H]
\centering
\includegraphics[width=0.95\textwidth]{image8.png}
\caption{Wskaźnik kredytów zagrożonych (NPL) w systemie bankowym.}
\label{fig:npl}
\end{figure}

Rysunek~\ref{fig:public-finances} przedstawia finanse publiczne. W fazie pre-BDP
rząd generuje niewielką nadwyżkę (VAT + CIT > wydatki
bazowe). Po wprowadzeniu BDP deficyt miesięczny skacze do ok. 130--150
mln PLN. Wydatki na BDP (200 mln/mies. = 100 000 osób × 2 000 PLN)
stanowią dominujący składnik budżetu. Dług skumulowany rośnie liniowo,
osiągając 12,4 mld PLN po 10 latach. W perspektywie MMT ten deficyt jest
funkcjonalny --- reprezentuje transfer netto do sektora prywatnego,
który finansuje zarówno konsumpcję (BDP), jak i inwestycje w
automatyzację (pośrednio, przez wzrost kosztów pracy wymuszający adopcję
AI).

\begin{figure}[H]
\centering
\includegraphics[width=0.95\textwidth]{image9.png}
\caption{Finanse publiczne: deficyt miesięczny i dług skumulowany.}
\label{fig:public-finances}
\end{figure}

\textbf{Podsumowanie symulacji.} Model SFC-ABM potwierdza i wzbogaca
analizę statyczną z sekcji 4.4. Kluczowe emergentne wnioski:

\textbf{(1) Timing:} Fala adopcji AI ma kształt S-krzywej z opóźnieniem
$\sim$10 miesięcy od szoku. Nie jest to natychmiastowe
przejście, lecz stopniowa dyfuzja hamowana przez bariery kompetencyjne i
dostęp do kredytu.

\textbf{(2) Overshooting monetarny:} NBP musi podnieść stopy do 20\%, by
opanować inflację. Cena stabilizacji cenowej to wyższy koszt kredytu,
który paradoksalnie przyspiesza automatyzację (pętla ② → ④).

\textbf{(3) Deflacja technologiczna: W końcowej fazie inflacja
przechodzi w deflację (−0,6\%), potwierdzając hipotezę z sekcji 3.3 o
endogenicznym mechanizmie stabilizacyjnym: automatyzacja generowana
przez BDP sama koryguje presję inflacyjną, którą BDP wywołało.}

\textbf{(4) Nierówność strukturalna:} 31\% populacji kończy symulację
poza rynkiem pracy. BDP zapewnia minimum egzystencjalne, ale nie
zapobiega stratyfikacji opisanej w sekcji 5.3 (tania egzystencja vs.
drogie aktywa).

\subsubsection{4.8. Analiza wrażliwości: kontrafaktyczny scenariusz bez
BDP}\label{analiza-wraux17cliwoux15bci-kontrafaktyczny-scenariusz-bez-bdp}

Dla weryfikacji roli BDP jako katalizatora automatyzacji przeprowadzono
analizę wrażliwości, porównując trzy scenariusze na identycznej
populacji 10 000 firm: (i) kontrafaktyczny --- BDP = 0 PLN, gdzie premia
za niepewność pozostaje wysoka przez cały
horyzont, odzwierciedlając brak presji kosztowej; (ii) bazowy --- BDP =
2 000 PLN/os./mies. (scenariusz główny z sekcji 4.7); (iii) eskalacyjny
--- BDP = 3 000 PLN/os./mies.

\begin{figure}[H]
\centering
\includegraphics[width=0.95\textwidth]{image17.png}
\caption{Monte Carlo SFC-ABM: 100 seedów × 3 scenariusze (pasma = 90\% CI). Sześć paneli pokazuje inflację, bezrobocie, adopcję, kurs walutowy, płace i dług publiczny.}
\label{fig:4-8}
\end{figure}

\begin{figure}[H]
\centering
\includegraphics[width=0.95\textwidth]{image18.png}
\caption{Bimodalność adopcji (BDP = 2 000 PLN), adopcja per sektor oraz przestrzeń fazowa adopcja × inflacja.}
\label{fig:4-8a}
\end{figure}

\begin{figure}[H]
\centering
\includegraphics[width=0.95\textwidth]{image19.png}
\caption{Nieliniowa odpowiedź: odwrócone U adopcji technologicznej i inflacji jako funkcja poziomu BDP. Słupki błędów = ±1$\sigma$ z Monte Carlo (N = 100).}
\label{fig:4-8b}
\end{figure}

\begin{figure}[H]
\centering
\includegraphics[width=0.95\textwidth]{image20.png}
\caption{Adopcja technologiczna per sektor (BDP = 2 000 PLN): BPO/SSC ($\sigma$=50) najszybsza, ochrona zdrowia ($\sigma$=2) najwolniejsza. Pasma = 90\% CI, N = 100 seedów.}
\label{fig:4-8c}
\end{figure}

\textbf{Tabela 4.1. Porównanie scenariuszy w miesiącu 120 (M120).}

\begin{figure}[H]
\centering
\includegraphics[width=0.95\textwidth]{image22.png}
\caption{Test bimodalności: (A) histogram adopcji BDP = 2 000 z dopasowaniem KDE i GMM, (B) selekcja modelu BIC (optimum K = 3), (C) porównanie gęstości 3 scenariuszy. Test Hartigana: p = 1,7 × 10⁻⁵.}
\label{fig:4-8d}
\end{figure}

\begin{figure}[H]
\centering
\includegraphics[width=0.95\textwidth]{image23.png}
\caption{Diagram bifurkacyjny: ciągły sweep BDP 0--5 000 PLN (30 seedów × 21 punktów = 630 symulacji). Panel A: rozrzut adopcji z linią średnią i pasmem ±1$\sigma$. Panel B: inflacja z przejściem od deflacji do hiperinflacji. Panel C: wariancja adopcji --- szczyt przy BDP = 2 000 PLN potwierdza punkt krytyczny. Panel D: bezrobocie.}
\label{fig:4-8e}
\end{figure}

Diagram bifurkacyjny (Rysunek~\ref{fig:4-8e}) ujawnia pełną topologię przestrzeni
parametrycznej. Wariancja adopcji (panel C) osiąga maksimum przy BDP =
2 000 PLN ($\sigma$ = 16,4\%), co jest klasyczną sygnaturą punktu krytycznego w
fizyce statystycznej (critical slowing down, Scheffer et al. (2009)).
Poniżej progu (BDP < 1 250), gospodarka jest zamrożona w
spirali deflacyjnej Fishera z bezrobociem przekraczającym 80\%. Powyżej
progu (BDP > 2 500), system stabilizuje się przy
$\sim$32\% adopcji niezależnie od poziomu transferu --- rosnąca
inflacja (do 52\% przy BDP = 5 000) blokuje transformację. Jedynie
wąskie okno 1 500--2 250 PLN generuje wysokie średnie adopcje
(\textgreater45\%), przy czym BDP = 1 750 PLN maksymalizuje średnią
adopcję (70,1\%), a BDP = 2 000 PLN generuje najwyższą wariancję
(bimodalność), lokując system dokładnie w punkcie krytycznym.

\begin{table}[H]
\centering
\caption{Porównanie scenariuszy w miesiącu 120 (M120). Wartości: średnia $\pm$ odch. std. z N=100 seedów Monte Carlo.}
\label{tab:porownanie-scenariuszy}
\begin{tabular}{lccc}
\toprule
\textbf{Metryka} & \textbf{Bez BDP} & \textbf{BDP = 2\,000} & \textbf{BDP = 3\,000} \\
\midrule
Inflacja (r/r) & $-22{,}6\%$ & $-13{,}4\%$ & $+19{,}4\%$ \\
Bezrobocie & $78{,}7\%$ & $39{,}6\%$ & $19{,}4\%$ \\
Auto + Hybryda & $12{,}9\%$ & $61{,}9\%$ & $32{,}8\%$ \\
Kurs PLN/EUR & $3{,}29$ & $4{,}66$ & $5{,}08$ \\
Płaca rynkowa & 4\,000 PLN & 5\,331 PLN & 6\,487 PLN \\
Dług publiczny & $-0{,}83$ mld & $12{,}58$ mld & $15{,}23$ mld \\
\bottomrule
\end{tabular}
\end{table}

Wyniki ujawniają nieliniową relację w kształcie odwróconego U pomiędzy
poziomem BDP a tempem adopcji technologicznej. W scenariuszu
kontrafaktycznym (BDP = 0) firmy zachowują opcję czekania --- premia za
niepewność hamuje inwestycje, a tylko
12,9\% ± 4,3\% firm przechodzi na model zautomatyzowany lub hybrydowy do
M120. Konsekwencją jest głęboka deflacja (−22,6\%) i masowe bezrobocie
(78,7\% ± 3,7\%). Należy podkreślić, że deflacja w scenariuszu bez BDP
ma charakter fundamentalnie odmienny od deflacji technologicznej w
scenariuszu bazowym (−0,6\%). Przy zaledwie 12,9\% ± 4,3\% automatyzacji
spadek cen nie wynika z nadpodaży generowanej przez AI, lecz z kolapsu
popytu zagregowanego: 79\% populacji bez pracy i bez transferu BDP
powoduje spiralę deflacyjną typu debt-deflation (Fisher \cite{fisher1933}). Jest to
recesyjna deflacja popytowa, nie produktywnościowa deflacja
technologiczna --- co potwierdza, że BDP jest konieczny nie tylko jako
katalizator automatyzacji, ale również jako stabilizator
makroekonomiczny zapobiegający depresji.

Scenariusz bazowy (BDP = 2 000 PLN) okazuje się optymalny: presja
kosztowa aktywuje mechanizm Sorosowskiej refleksyjności, zmuszając
61,9\% ± 16,4\% firm do adopcji AI/hybrydy (rozkład bimodalny), przy
jednoczesnej stabilizacji inflacją końcową średnio na poziomie −13,4\% ±
10,5\% (wysoka wariancja wynika z bimodalności adopcji). Bezrobocie
strukturalne (39,6\% ± 13,1\%) jest paradoksalnie niższe niż bez BDP,
ponieważ szybsza automatyzacja generuje wyższą produktywność i
stabilizuje sektor bankowy (NPL < 2\%).

Scenariusz eskalacyjny (BDP = 3 000 PLN) ilustruje patologię nadmiernej
stymulacji. Presja inflacyjna (19,4\% ± 2,3\%) wymusza podniesienie stóp
do limitu (25\%), co blokuje dostęp do kredytu inwestycyjnego i
paradoksalnie spowalnia automatyzację (jedynie 32,8\% ± 2,1\%). Niskie
bezrobocie (19,4\% ± 1,2\%) jest artefaktem wysokiego nominalnego
transferu, który podtrzymuje konsumpcję mimo braku transformacji
technologicznej --- gospodarki „zombie" utrzymywanej przez druk
pieniądza.

Analiza wrażliwości potwierdza zatem centralną tezę pracy: istnieje
optymalny poziom BDP, który maksymalizuje transformację technologiczną
przy akceptowalnym koszcie inflacyjnym. Zarówno brak BDP (stagnacja z
opcją czekania), jak i nadmierny BDP (inflacyjna blokada kredytowa)
prowadzą do gorszych wyników. Jest to wynik spójny z logiką
nieergodyczności Ole Petersa --- optymalny szok musi być wystarczająco
silny, by zneutralizować premię za niepewność, ale nie tak silny, by
zdestabilizować kanał monetarny. Co więcej, analiza Monte Carlo ujawniła
fundamentalny wynik z perspektywy ekonomii złożoności: rozkład adopcji
dla BDP = 2 000 PLN jest bimodalny. Dekompozycja GMM (K = 3) wykazuje
trzy stany atraktorowe: $\sim$59\% realizacji (N = 100 seedów)
przechodzi do atraktora wysokiej adopcji ($\mu$ = 73,2\%),
$\sim$21\% osiada w stanie niskiej adopcji ($\mu$ = 34,2\%), a
$\sim$19\% pozostaje w stanie pośrednim ($\mu$ = 57,6\%). Ta
wielomodalność jest sygnaturą systemu bliskiego punktu krytycznego
(phase transition) i potwierdza fundamentalną nieergodyczność
transformacji: średni wynik (62\%) nie odpowiada żadnej typowej
realizacji. Scenariusze BDP = 0 i BDP = 3 000 PLN wykazują natomiast
rozkłady unimodalne o niskiej wariancji ($\sigma$ < 2,1\%), co
oznacza, że jedynie optymalny poziom BDP generuje zachowanie krytyczne.
Formalny test statystyczny potwierdza tę obserwację: test Hartigana (dip
test) odrzuca hipotezę unimodalności dla BDP = 2 000 PLN z p = 1,7 ×
10⁻⁵ (***), podczas gdy BDP = 0 (p = 0,75) i BDP = 3 000 (p = 0,97) są
jednoznacznie unimodalne. Selekcja modelu Gaussian Mixture via BIC
wskazuje na K = 3 komponenty: dominujący klaster wysokiej adopcji ($\mu$ =
73,2\%, waga 0,59), klaster niskiej adopcji ($\mu$ = 34,2\%, waga 0,21) i
przejściowa grupa pośrednia ($\mu$ = 57,6\%, waga 0,19).

\subsubsection{4.9. Analiza dobrostanu (welfare
analysis)}\label{analiza-dobrostanu-welfare-analysis}

Sama analiza makroagregatów (inflacja, bezrobocie, adopcja) nie
wystarcza do oceny skutków polityki BDP. Konieczna jest miara dobrostanu
uwzględniająca zarówno poziom, jak i rozkład dochodów. W tym celu
obliczono dwa wskaźniki welfare na podstawie danych z trzech scenariuszy
symulacji: (i) realną konsumpcję per capita (nominalna konsumpcja MPC ×
dochód, deflowana skumulowanym indeksem cen) oraz (ii) współczynnik
Giniego dla dwóch klas dochodowych --- zatrudnionych (płaca + BDP) i
niezatrudnionych (wyłącznie BDP lub zero w scenariuszu
kontrafaktycznym).

\begin{figure}[H]
\centering
\includegraphics[width=0.95\textwidth]{image21.png}
\caption{Analiza dobrostanu Monte Carlo (N = 100 seedów × 3 scenariusze): realna konsumpcja per capita, współczynnik Giniego, tradeoff równość--konsumpcja, porównanie wielowymiarowe.}
\label{fig:4-9}
\end{figure}

Wyniki ujawniają pozorny paradoks: scenariusz bez BDP wykazuje realną
konsumpcję per capita na poziomie 2 311 ± 329 PLN/mies. w M120 ---
niższą niż scenariusz bazowy (5 950 ± 2 024 PLN dla BDP = 2 000), lecz
wyższą niż scenariusz eskalacyjny (1 570 ± 205 PLN dla BDP = 3 000).
Kluczowe jest jednak to, czego średnia nie pokazuje: przy 78,7\%
bezrobociu i zerowym transferze, 78 700 osób (z populacji 100 000) ma
dosłownie zerowy dochód. Współczynnik Giniego 0,80 potwierdza
ekstremalną nierówność --- jest to scenariusz deflacyjnej depresji, nie
dobrobytu. Zjawisko to ilustruje klasyczną „pułapkę średniej'' (GDP
fallacy): umiarkowany agregatowy wskaźnik maskuje katastrofę humanitarną
na poziomie indywidualnym.

Scenariusz bazowy (BDP = 2 000 PLN) oferuje najlepszy kompromis:
najwyższa realna konsumpcja (5 950 ± 2 024 PLN) przy Gini = 0,20 i ---
co najważniejsze --- gwarantowanym dochodzie minimalnym 2 000 PLN/mies.
dla każdego obywatela (income floor). Żaden uczestnik gospodarki nie
zostaje pozbawiony środków do życia, a jednocześnie presja kosztowa
maksymalizuje transformację technologiczną (62\% firm ($\mu$ z Monte
Carlo)). Scenariusz eskalacyjny (BDP = 3 000 PLN) minimalizuje
nierówność (Gini = 0,10) i zapewnia najwyższy floor dochodowy, lecz za
cenę destrukcji siły nabywczej: inflacja 19,4\% redukuje realną
konsumpcję do zaledwie 1 570 ± 205 PLN --- paradoks nominalnego bogactwa
przy realnej biedzie.

Analiza dobrostanu potwierdza zatem istnienie trójstronnego tradeoffu:
efektywność transformacyjna (automatyzacja), równość dochodowa (Gini) i
realna siła nabywcza (konsumpcja) nie mogą być jednocześnie
maksymalizowane. BDP = 2 000 PLN lokuje się w punkcie Pareto-optymalnym
tego trójkąta: nie minimalizuje nierówności (to robi BDP = 3 000 ---
kosztem destrukcji siły nabywczej), nie eliminuje biedy (to robi jedynie
wyższy floor dochodowy), lecz jednocześnie zapewnia najszybszą
transformację technologiczną, akceptowalny poziom nierówności i
niezerowy floor dochodowy. W terminologii ekonomii złożoności jest to
atraktor optymalny w przestrzeni wielokryterialnej --- stabilny
kompromis emergentny, niedostępny w modelach jednowymiarowej
optymalizacji. Analiza Monte Carlo (N = 100) potwierdza ten wniosek:
mediana Giniego dla BDP = 2 000 wynosi 0,20, podczas gdy BDP = 0 daje
0,80 (ekstremalna nierówność --- spirala deflacyjna Fishera), a BDP = 3
000 --- zaledwie 0,10 (pozorna równość przy destrukcji siły nabywczej).

\subsubsection{4.10. Kalibracja strukturalna GUS 2024: model
6-sektorowy}\label{kalibracja-strukturalna-gus-2024-model-6-sektorowy}

Dotychczasowa analiza (sekcje 4.7--4.9) opierała się na modelu
stylizowanym z równymi udziałami sektorowymi (30/30/25/15\%), który
izoluje mechanizm Paradoksu Akceleracji w czystej formie. Sekcja
niniejsza testuje robustność wyników poprzez kalibrację struktury
sektorowej modelu do danych GUS/NBP 2024, co ujawnia, jak kompozycja
polskiej gospodarki mediuje odkryte mechanizmy.

Kalibracja obejmuje: (i) rozszerzenie modelu o dwa nowe sektory ---
budżetówkę (22\% zatrudnienia, $\sigma$ = 1) i rolnictwo (8\%, $\sigma$ = 3); (ii)
dostosowanie udziałów sektorowych do struktury BAEL: BPO/SSC 3\%,
przemysł 16\%, handel i usługi 45\%, ochrona zdrowia 6\%, sektor
publiczny 22\%, rolnictwo 8\%; (iii) wprowadzenie sektorowych mnożników
płacowych odzwierciedlających dane GUS o wynagrodzeniach (BPO ×1,35 vs
rolnictwo ×0,67 średniej krajowej); (iv) aktualizację parametrów
makroekonomicznych: płaca bazowa 8 266 PLN (GUS 2024), stopa NBP 5,75\%,
spread kredytowy 1,5\%, kurs PLN/EUR 4,33.

Wyniki Monte Carlo (N = 100 seedów × 3 scenariusze) ujawniają
fundamentalnie odmienną dynamikę makroekonomiczną:

Tabela 4.2. Wyniki modelu skalibrowanego GUS 2024 (M120, N = 100)

BDP = 0: adopcja 13,9\% ± 2,9\%, inflacja −9,0\% ± 4,2\%, bezrobocie
8,6\% ± 2,4\%\\
BDP = 2 000: adopcja 13,1\% ± 1,7\%, inflacja +17,9\% ± 1,8\%,
bezrobocie 8,5\% ± 0,9\%\\
BDP = 3 000: adopcja 12,1\% ± 1,2\%, inflacja +33,8\% ± 1,7\%,
bezrobocie 8,1\% ± 0,7\%

\begin{figure}[H]
\centering
\includegraphics[width=0.95\textwidth]{image25.png}
\caption{Adopcja technologiczna per sektor i agregaty makro --- model GUS 2024 (N = 100).}
\label{fig:4-10a}
\end{figure}

Kluczowym odkryciem jest sektorowa heterogeniczność Paradoksu
Akceleracji. Sektor BPO/SSC ($\sigma$ = 50, stanowiący zaledwie 3\%
zatrudnienia) wykazuje silne przyspieszenie: adopcja rośnie z 63,5\%
(BDP = 0) do 84,5\% (BDP = 2 000), co stanowi przyrost +21pp ---
potwierdzając mechanizm CES opisany w sekcji 3.1. Jednocześnie sektor
przemysłowy ($\sigma$ = 10, 16\% zatrudnienia) wykazuje odwrotną reakcję:
adopcja spada z 45,0\% do 31,6\% (−13,4pp). Mechanizmem jest kanał
kredytowy: BDP generuje inflację (+17,9\%), na którą NBP reaguje
agresywną regułą Taylora, podnosząc stopy procentowe. Wyższy koszt
kredytu blokuje kapitałochłonną automatyzację w przemyśle
(mnożnik CAPEX = 1,12), podczas gdy BPO --- z niskim kosztem
wdrożenia AI (mnożnik CAPEX = 0,70) --- jest odporny na ten kanał.

\begin{figure}[H]
\centering
\includegraphics[width=0.95\textwidth]{image24.png}
\caption{Podwójny Paradoks: BDP przyspiesza BPO/SSC (+21pp), hamuje przemysł (−13pp) przez kanał kredytowy.}
\label{fig:4-10b}
\end{figure}

Zjawisko to można określić jako Podwójny Paradoks Akceleracji: BDP
jednocześnie przyspiesza automatyzację w sektorach o wysokiej
elastyczności substytucji i niskim koszcie wdrożenia (BPO, $\sigma$ = 50), a
hamuje ją w sektorach o niższej elastyczności i wyższych kosztach
kapitałowych (przemysł, $\sigma$ = 10). Trzy sektory --- handel/usługi (45\%),
sektor publiczny (22\%) i rolnictwo (8\%) --- stanowiące łącznie 75\%
gospodarki, pozostają praktycznie odporne na BDP w wymiarze
transformacyjnym (adopcja 0,2--11,7\% niezależnie od poziomu BDP).

Implikacje dla polityki gospodarczej są istotne. W gospodarce o
strukturze polskiej, gdzie sektory podatne na automatyzację (BPO +
przemysł) stanowią zaledwie 19\% zatrudnienia, zagregowany efekt
transformacyjny BDP jest ograniczony ($\sim$13\% niezależnie od
poziomu BDP). Dominującym efektem makroekonomicznym staje się kanał
inflacyjny: BDP = 2 000 PLN generuje inflację +17,9\%, a BDP = 3 000 PLN
aż +33,8\%. Sugeruje to, że w gospodarce zdominowanej przez usługi i
sektor publiczny, BDP jest przede wszystkim narzędziem redystrybucyjnym,
a nie transformacyjnym --- a jego potencjał transformacyjny koncentruje
się w wąskim segmencie gospodarki opartej na wiedzy.

Porównanie modelu stylizowanego (sekcja 4.7) z modelem GUS ujawnia zatem
komplementarną perspektywę: model stylizowany identyfikuje mechanizm
(Paradoks Akceleracji, przejście fazowe), podczas gdy model skalibrowany
pokazuje, jak struktura realna mediuje ten mechanizm. Oba wyniki są
spójne --- różnią się zakresem, nie kierunkiem oddziaływania.

\subsubsection{4.11. Ograniczenia modelu i kierunki dalszych
badań}\label{ograniczenia-modelu-i-kierunki-dalszych-badaux144}

Prezentowany model SFC-ABM, mimo spójności bilansowej i emergentnych
wyników, podlega istotnym ograniczeniom, które należy jawnie
zidentyfikować.

Po pierwsze, mimo rozszerzenia o Monte Carlo (N = 100), sieć
Wattsa-Strogatz i 6 sektorów (w wersji GUS 2024), model nadal zakłada
jednorodny towar i jednorodne gospodarstwa domowe (brak
heterogeniczności MPC, kwalifikacji czy preferencji konsumenckich).
Rozszerzenie o pełną heterogeniczność agentów-konsumentów (Dosi et al.
2010, Dawid et al. \cite{dawid2018}) stanowi naturalne uzupełnienie. Po drugie,
topologia sieciowa (Watts-Strogatz) jest statyczna (choć model
uwzględnia miękki floor deflacyjny modelujący sztywność cenową w dół ---
Bewley (1999)); w przyszłych iteracjach warto dodać endogeniczną ewolucję
sieci (rewiring adaptacyjny), co pozwoliłoby modelować powstawanie i
zanik łańcuchów dostaw w odpowiedzi na szok BDP.

Po trzecie, zagregowany sektor gospodarstw domowych nie uwzględnia
zróżnicowania reakcji na BDP: pracownicy wysoko wykwalifikowani mogą nie
reagować na transfer, podczas gdy pracownicy niskokwalifikowani mogą
masowo wycofać się z rynku pracy. Wprowadzenie heterogenicznych
agentów-konsumentów z różnym MPC, preferencjami i kwalifikacjami stanowi
naturalny kierunek rozszerzenia (por. Dosi et al. \cite{dosi2010}; Dawid et al.,
2018).

Po czwarte, analiza wrażliwości obejmuje wyłącznie poziom BDP. Przyszłe
badania powinny eksplorować przestrzeń parametryczną szerzej:
współczynniki reguły Taylora, próg gotowości cyfrowej, struktura bankowa
(CAR, spread kredytowy) oraz alternatywne topologie sieciowe (scale-free
Barabási-Albert vs. small-world Watts-Strogatz).

\setcounter{mychapter}{5}\setcounter{figure}{0}\setcounter{table}{0}
\section{ROZDZIAŁ 5}
\label{rozdzial-5}

\subsection{Implikacje Strategiczne i Geopolityczne: Zarządzanie w warunkach „Podwójnego Szoku''}
\label{implikacje-strategiczne-i-geopolityczne}

Przeprowadzona symulacja agentowa (Rozdział 4) wykazała, że BDP wprowadza gospodarkę w fazę przejściową charakteryzującą się ekstremalną zmiennością: skok inflacji do 13,7\%, deprecjacja waluty, wzrost bezrobocia strukturalnego do 31\% oraz falę bankructw (1,6\% firm). Z perspektywy strategii zarządczej i polityki publicznej kluczowe jest zrozumienie, jak podmioty gospodarcze powinny nawigować przez ten „podwójny szok'' (płacowy i monetarny) oraz jakie są implikacje geopolityczne tego procesu dla Polski jako małej gospodarki otwartej.

\subsubsection{5.1. Nowa definicja ryzyka: Praca ludzka jako „Aktywo Toksyczne''}
\label{nowa-definicja-ryzyka-praca-ludzka-jako-aktywo-toksyczne}

W klasycznej teorii zarządzania ryzykiem, kapitał ludzki jest traktowany jako zasób strategiczny -- źródło przewagi konkurencyjnej, innowacji i adaptacyjności. Model przedstawiony w niniejszej pracy wymusza fundamentalną rewizję tego paradygmatu w kontekście BDP.

W warunkach szoku BDP, praca ludzka przestaje być aktywem strategicznym, a staje się \textbf{aktywem toksycznym} -- źródłem strukturalnego ryzyka ruiny. Dzieje się tak z trzech powodów:

\begin{enumerate}
\item \textbf{Sztywność kosztowa:} BDP podnosi płacę rezerwową nieodwracalnie. W przeciwieństwie do aktywów finansowych, których wartość może spadać, koszt pracy ma charakter asymetryczny -- może tylko rosnąć.
\item \textbf{Ryzyko regulacyjne:} Państwo z BDP staje się de facto gwarantem płacy minimalnej. Firmy tracą zdolność do elastycznego dostosowywania wynagrodzeń w cyklu koniunkturalnym.
\item \textbf{Efekt substytucji:} Przy wysokiej elastyczności substytucji AI ($\sigma \rightarrow \infty$), każdy dodatkowy pracownik podnosi ryzyko utraty konkurencyjności względem firm zautomatyzowanych.
\end{enumerate}

\textbf{Implikacja strategiczna:} Firmy działające w sektorach o wysokiej podatności na automatyzację (BPO/SSC, finanse, obsługa klienta) muszą przebudować swoje portfele ryzyka, traktując zatrudnienie jako zobowiązanie warunkowe (contingent liability) podobne do długu podporządkowanego. Zarządy powinny opracować plany transformacji technologicznej już w fazie pre-BDP, aby uniknąć bariery pochłaniającej.

\subsubsection{5.2. Strategia „Capital Deepening'' i ryzyko oligopolizacji}
\label{strategia-capital-deepening-i-ryzyko-oligopolizacji}

Symulacja wykazała, że jedynie 4\% firm osiąga pełną automatyzację (Automated), podczas gdy 54\% pozostaje w stanie hybrydowym. Ta asymetria wynika z różnic w zdolności kredytowej (creditworthiness) i gotowości cyfrowej. Rodzi to pytanie o długoterminową strukturę rynku.

Firmy, które jako pierwsze osiągną pełną automatyzację, uzyskają gigantyczną przewagę kosztową (EBITDA 46,7 mln PLN vs. strata 1 mln PLN w scenariuszu biernym). Ta przewaga pozwoli im na:

\begin{enumerate}
\item \textbf{Ekspansję poprzez wykup:} Przejęcie bankrutujących konkurentów po cenach likwidacyjnych.
\item \textbf{Dumpingową politykę cenową:} Wykorzystanie zerowego kosztu krańcowego AI do wypierania firm hybrydowych z rynku.
\item \textbf{Akumulację kapitału:} Reinwestowanie gigantycznych zysków w dalszy rozwój AI (self-reinforcing loop).
\end{enumerate}

\textbf{Ryzyko oligopolizacji:} W długim okresie (15--20 lat po wprowadzeniu BDP), gospodarka może ewoluować w stronę struktury oligopolistycznej, gdzie 3--5 „super-firm'' z pełną automatyzacją kontroluje większość rynku w każdym sektorze. To zjawisko jest obserwowane już dziś w sektorze technologicznym (GAFA -- Google, Apple, Facebook, Amazon) i może się rozprzestrzenić na wszystkie branże podatne na automatyzację.

\textbf{Implikacja dla polityki konkurencji:} Regulatorzy muszą przygotować się na erozję tradycyjnych barier wejścia. W świecie AI, bariery nie są technologiczne (model można kupić), lecz kapitałowe (CAPEX) i kompetencyjne (gotowość cyfrowa). Może to wymagać nowych instrumentów antymonopolowych, takich jak podatek od automatyzacji lub obowiązkowe licencjonowanie algorytmów.

\subsubsection{5.3. Dychotomia rynku konsumenckiego: „Tania egzystencja vs. Drogie aktywa''}
\label{dychotomia-rynku-konsumenckiego}

Kluczowym paradoksem modelu jest współistnienie deflacji technologicznej (ceny dóbr masowych spadają do zera) z rosnącym bezrobociem strukturalnym. Jak w takim świecie wygląda rynek konsumencki?

Symulacja sugeruje powstanie \textbf{dychotomii konsumpcyjnej}:

\begin{enumerate}
\item \textbf{Tania egzystencja:} Dobra i usługi wytwarzane przez AI (rozrywka cyfrowa, aplikacje, usługi B2C) stają się praktycznie darmowe. BDP zapewnia minimum pozwalające na konsumpcję tych dóbr.
\item \textbf{Drogie aktywa:} Nieruchomości, grunty, dzieła sztuki, produkty luksusowe -- wszystko, czego nie można wyprodukować przez AI -- drożeją wykładniczo. Kapitał koncentruje się w rękach właścicieli firm zautomatyzowanych.
\end{enumerate}

\textbf{Implikacja społeczna:} BDP w połączeniu z deflacją technologiczną tworzy społeczeństwo „komfortowej stagnacji'' -- ludzie mają dostęp do tanich dóbr egzystencjalnych (jedzenie, rozrywka, komunikacja), ale są odcięci od możliwości akumulacji majątku i awansu społecznego. Może to prowadzić do wzrostu nierówności majątkowych mimo spadku nierówności konsumpcyjnych.

\subsubsection{5.4. Analiza kontrfaktyczna: BDP w reżimie strefy Euro -- od inflacji do niewypłacalności}
\label{analiza-kontrfaktyczna-bdp-w-rezimie-strefy-euro}

Dotychczasowa analiza zakładała, że Polska posiada własny bank centralny (NBP) i płynny kurs walutowy (PLN). To założenie jest kluczowe dla mechanizmu endogenicznej deflacji technologicznej. Co by się stało, gdyby Polska była w strefie Euro?

W strefie Euro, Polska traciłaby autonomię monetarną. Europejski Bank Centralny (EBC) ustala stopy procentowe dla całej strefy, optymalizując politykę dla gospodarek rdzenia (Niemcy, Francja), a nie peryferii (Polska, Hiszpania). Wprowadzenie BDP w Polsce wywołałoby:

\begin{enumerate}
\item \textbf{Lokalną presję inflacyjną:} Wzrost popytu w Polsce, ale bez możliwości podniesienia stóp przez NBP.
\item \textbf{Ucieczkę kapitału:} Inwestorzy zagraniczni, widząc rosnącą inflację i brak reakcji monetarnej, wyprzedają polskie obligacje.
\item \textbf{Kryzys fiskalny:} Polska, nie mogąc sfinansować deficytu poprzez kreację własnej waluty, musi pożyczać na rynkach po rosnących spreadach.
\item \textbf{Brak mechanizmu wymuszającego automatyzację:} Bez szoku odsetkowego (bo stopy są ustalone przez EBC dla całej strefy), firmy nie mają presji do automatyzacji. BDP prowadzi jedynie do inflacji bez transformacji technologicznej.
\end{enumerate}

\textbf{Wniosek geopolityczny:} Wprowadzenie BDP w strefie Euro jest potencjalnie destrukcyjne dla gospodarek peryferyjnych. Paradoksalnie, to gospodarki z własną walutą (Polska, Czechy, Węgry) mają lepsze warunki do przeprowadzenia transformacji BDP-AI, ponieważ dysponują narzędziem stóp procentowych jako filtrem ewolucyjnym dla firm.

\subsubsection{5.5. Złudzenie planowania: Dlaczego polityka dotacyjna nie zastąpi szoku cenowego?}
\label{zludzenie-planowania-dlaczego-polityka-dotacyjna}

Krytycy przedstawionej tezy mogą argumentować: \emph{„Zamiast wprowadzać BDP i ryzykować chaos, rząd może po prostu dotować automatyzację, oferując firmom preferencyjne kredyty i ulgi podatkowe na zakup AI.''}

To podejście ignoruje fundamentalny mechanizm nieergodyczny i ryzyko pokusy nadużycia (moral hazard).

\paragraph{5.5.1. Problem Pokusy Nadużycia (Moral Hazard) i Alokacji Kapitału}
\label{problem-pokusy-naduzycia}

W scenariuszu dotacyjnym, firmy otrzymują państwowy kapitał bez presji kosztowej. To tworzy dwa problemy:

\begin{enumerate}
\item \textbf{Brak filtra jakości:} W modelu rynkowym (BDP + wysokie stopy), tylko firmy o wysokiej gotowości cyfrowej i silnym zarządzaniu przetrwają transformację. W modelu dotacyjnym, kapitał trafia również do firm nieefektywnych (zombie firms), które nie mają rzeczywistej zdolności do implementacji AI.
\item \textbf{Pokusa nadużycia:} Firmy mogą pozyskiwać dotacje, inwestować w nieoptymalne rozwiązania technologiczne (bo są tańsze) lub nawet marnotrawić kapitał, wiedząc, że nie poniosą pełnego ryzyka ruiny.
\end{enumerate}

\textbf{Dowód empiryczny:} Programy dotacyjne Unii Europejskiej na cyfryzację (Digital Europe Programme, budżet 7,5 mld EUR) pokazują niską efektywność -- większość środków trafia do konsultantów i projektów pilotażowych, które nigdy nie są skalowane.

\paragraph{5.5.2. Ochrona firm „Zombie'' vs. Twórcza Destrukcja}
\label{ochrona-firm-zombie-vs-tworcza-destrukcja}

W modelu prezentowanym w tej pracy, 1,6\% firm bankrutuje (bariera pochłaniająca). Choć brzmi to dramatycznie, jest to klasyczny mechanizm \textbf{twórczej destrukcji} Schumpetera. Firmy, które bankrutują, to te o niskiej gotowości cyfrowej, słabym zarządzaniu i przestarzałych modelach biznesowych. Ich wyjście z rynku uwalnia zasoby (kapitał, pracowników) dla firm innowacyjnych.

W scenariuszu dotacyjnym, państwo sztucznie podtrzymuje egzystencję firm zombie, prowadząc do zjawiska znanego z japońskiej „straconej dekady'' (1991--2001): gospodarka stagnuje, innowacje są hamowane, a kapitał jest zamrożony w nieproduktywnych strukturach.

\textbf{Konkluzja:} Polityka dotacyjna bez presji rynkowej prowadzi do złudzeń planowania centralnego. Jedynie mechanizm cenowy (droga praca + drogi pieniądz) jest skutecznym filtrem, który wymusza rzeczywistą transformację, eliminując nieefektywne podmioty i alokując kapitał tam, gdzie jest produktywny.

\setcounter{mychapter}{6}\setcounter{figure}{0}\setcounter{table}{0}
\section{ZAKOŃCZENIE}
\label{zakonczenie}

\subsection{Synteza: Od Błędu Ergodyczności do Imperatywu Automatyzacji}
\label{synteza-od-bledu-ergodycznosci}

Niniejsza praca podjęła ambitne zadanie przewartościowania dotychczasowego dyskursu na temat Bezwarunkowego Dochodu Podstawowego (BDP). W przeciwieństwie do dominującej narracji, która traktuje BDP jako narzędzie osłonowe wobec nieuchronnej fali bezrobocia technologicznego, wykazano, że relacja przyczynowa jest odwrotna: \textbf{BDP nie jest skutkiem automatyzacji, lecz jej pierwotną przyczyną}.

Fundamentem tej tezy jest odrzucenie założenia o ergodyczności procesów gospodarczych -- kluczowego błędu, który przenika modele głównego nurtu ekonomii, w tym raporty MFW i modele DSGE. Praca udowodniła, że w rzeczywistości menedżerskiej czas jest zasobem krytycznym, a przedsiębiorstwa działają w warunkach \textbf{nieergodycznych} -- gdzie istnienie bariery pochłaniającej (bankructwo) zmienia naturę ryzyka z optymalizacyjnej na egzystencjalną.

Wprowadzenie BDP w małej gospodarce otwartej (Polska) wywołuje podwójny szok: \textbf{płacowy} (wzrost płacy rezerwowej) oraz \textbf{monetarny} (reakcja stóp procentowych na inflację). W obliczu tego szoku, tradycyjne modele biznesowe oparte na pracy ludzkiej stają się nierentowne matematycznie. Jedyną strategią przetrwania jest skokowa, nieliniowa transformacja w stronę aktywów o zerowym koszcie krańcowym -- czyli sztucznej inteligencji.

Kluczowym wkładem teoretycznym pracy jest wprowadzenie koncepcji \textbf{„Endogenicznej Deflacji Technologicznej''}. Wykazano, że masowe inwestycje w AI, wymuszone przez presję kosztową BDP, prowadzą do tak gigantycznego wzrostu produktywności, że gospodarka „wchłania'' nadmiarowy pieniądz poprzez lawinę tanich dóbr i usług. Tym samym, AI przejmuje w modelu makroekonomicznym rolę podatków, rozwiązując dylemat inflacyjny MMT bez konieczności drastycznych podwyżek obciążeń fiskalnych.

Przeprowadzona symulacja agentowa SFC-ABM (10 000 firm, 100 replikacji Monte Carlo) potwierdziła predykcje teoretyczne, wykazując emergentną dynamikę: S-krzywą adopcji technologii, szczyt inflacji 13,7\% opanowany regułą Taylora oraz przejście w deflację technologiczną w fazie końcowej. Co kluczowe, kalibracja strukturalna do danych GUS 2024 ujawniła \textbf{Podwójny Paradoks Akceleracji}: BDP jednocześnie przyspiesza automatyzację w sektorach o wysokiej elastyczności substytucji (BPO/SSC), a hamuje ją w przemyśle przez kanał kredytowy -- przy czym 75\% gospodarki pozostaje odporne na transformację.

Analiza kontrfaktyczna wykazała, że powodzenie transformacji BDP-AI jest ściśle uzależnione od posiadania suwerennej waluty i autonomicznego banku centralnego. W reżimie strefy Euro, pozbawionej mechanizmu stóp procentowych jako filtra ewolucyjnego, BDP prowadzi do ryzyka niewypłacalności państwa bez modernizacji technologicznej.

Z perspektywy zarządczej, praca udowadnia, że w erze BDP praca ludzka przestaje być aktywem strategicznym, a staje się \textbf{aktywem toksycznym}. Firmy muszą przebudować swoje portfele ryzyka, traktując zatrudnienie jako zobowiązanie warunkowe. Z perspektywy polityki publicznej, praca wykazuje, że polityka dotacyjna nie może zastąpić mechanizmu cenowego -- jedynie rzeczywisty szok kosztowy wymusza efektywną transformację poprzez mechanizm twórczej destrukcji.

Najważniejszym przesłaniem pracy jest to, że \textbf{drożyzna nie jest problemem -- jest rozwiązaniem}. Wysoki koszt pracy i pieniądza nie hamuje rozwoju gospodarczego, lecz działa jako katalizator niezbędnej transformacji technologicznej. Paradoksalnie, to właśnie mechanizm rynkowy -- a nie planowanie centralne -- jest w stanie przeprowadzić gospodarkę przez fazę przejściową, chroniąc wartość waluty poprzez wymuszenie efektywności.

Niniejsza praca otwiera nowe kierunki badań na styku ekonomii złożoności, teorii monetarnej i strategii zarządzania. Wymaga dalszej eksploracji, szczególnie w zakresie heterogenicznych gospodarstw domowych, endogenicznych innowacji AI oraz dynamiki handlu międzynarodowego. Niemniej jednak, już w obecnej formie dostarcza fundamentalnie nowej perspektywy na debatę o przyszłości pracy, pieniądza i technologii w XXI wieku.

\textbf{Konkluzja końcowa:} BDP nie jest programem społecznym. Jest mechanizmem transformacji technologicznej -- pożądanym lub niepożądanym, zależnie od gotowości gospodarki. Dla Polski, z niskim kapitałem na pracownika i suwerenną walutą, może to być szansa na skokowe przyspieszenie modernizacji. Dla gospodarek strefy Euro, jest to potencjalna pułapka fiskalną. Kluczem jest zrozumienie, że technologia nie jest zmienną egzogeniczną -- jest endogeniczną reakcją na strukturę kosztów, którą kształtuje polityka publiczna.

\printbibliography

\end{document}
